\section{Reflection}

\subsection*{Risk Assessment}

\paragraph*{Difficulty with blockchain development}
As expected there were several difficulties with blockchain development:

\begin{itemize}
  \item A decent amount of the documentation is minimal or outdated. As blockchain development is a smaller field there are little in the ways of forum postings.
  \item Some of the libraries used had bugs in them that hindered development. For example, I could deploy the smart contract to the Sepolia testnet from Remix easily but not through Geth. 
\end{itemize}

\newparagraph
Despite the problems mentioned above, I was able to complete the blockchain aspect of the project successfully. Effective use of sprint planning meant that I had considered this to be a potential issue and had allocated time to being able to deal with it.

\paragraph*{The application is not finished}
Section~\ref{sec:sprints} shows that not all of the requirements were finished and Section~\ref{sec:design-lim} discusses the limitations with my current implementation. It was expected that this project would not be entirely finished but a lot of the major requirements were met.
\x
The application ended up being incredibly large and took a lot more time to implement and test than expected. Table~\ref{tab:cloc} shows a breakdown of the source code. Agile development could only help me so much so by having a hard cut-off point for the implementation I was still left with plenty of time to complete the report.

\begin{longtable}{ r r r r }
  \toprule
  \textbf{Language} & \textbf{Files} & \textbf{Comments} & \textbf{Code}
  \\\midrule\midrule
  Go
  & 45
  & 829
  & 5006
  \\
  Go Tests
  & 26
  & 665
  & 2836
  \\
  Vue.js Components
  & 17
  & 141
  & 2414
  \\
  JavaScript
  & 19
  & 88
  & 377
  \\
  Solidity
  & 1
  & 33
  & 46
  \\\bottomrule\bottomrule
  \caption{The lines of code written for this project calculated using CLOC~\cite{noauthor_aldanialcloc_nodate}}
  \label{tab:cloc}
\end{longtable}

\paragraph*{Lack of large-scale testing infrastructure}
Testing distributed applications is challenging due to many factors, such as having to source homogenous devices, having access to networks that would allow me host a peer, and more. However, several useful results were produced from the benchmarks (Section~\ref{sec:benchmark}) and acceptance tests (Section~\ref{sec:acc-tests}) that can be used to improve the application. If this project were to move forward then testing it on a large-scale would be essential.

\subsection*{Using Agile Development}

The use of an agile methodology alongside test-driven development meant I could incrementally expand the scope of my application and ensure that existing code was tested sufficiently. Separating my implementation into three sprints benefitted me by:

\begin{itemize}
  \item having a smaller set of requirements to focus on at once helped me to feel less overwhelmed,
  \item working on the most important aspects first to ensure I was able to produce a minimum viable product, and
  \item taking time in-between sprints to take a break from the project and prepare for the next sprint.
\end{itemize}

\newparagraph
On top of this, using a Gantt chart (Section~\ref{fig:gantt-chart-1}) allowed me to have a clear overview of my project timeline so I could realistically gage just how much time I would need for each section.