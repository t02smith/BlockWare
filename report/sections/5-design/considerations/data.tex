% chktex-file 24
% chktex-file 8

\subsection{Data Types}
\label{subsec:design-data}

Table~\ref{tab:data} discusses the different types of data we are going to need to store and where they should be stored based upon their properties.


\begin{longtable}{ p{.12\textwidth} p{.1\textwidth} p{.1\textwidth} p{.63\textwidth} }
  \toprule
  \textbf{Data} & \textbf{Size} & \textbf{Location} & \textbf{Explanation}\\
  \midrule\midrule
  Game Metadata\newline\reqref{F-M1}
  & \small100 -- \newline200B
  & Ethereum
  & \small This data is the minimal set of information required for the unique identification of each game. See Section~\ref{subsubsec:eth-data}.

  \vspace{1mm}
  \small This data is appropriate to store on Ethereum as it is public, small in size, and essential to the correct functioning of the application as all users will need to be able to discover all games. 
  \x
  Game Hash Tree\newline\reqref{F-M12}
  & \~ \small15KB
  & IPFS
  & \small This will be the compressed hash tree that will allow users to identify and verify the blocks of data they need to download for a game. The user will download this immediately after purchasing the game.

  \vspace{1mm}
  \small This data is public but its size will make it costly to store on Ethereum at a large scale and given that only a subset of users will actually ever want access to it, it would be wrong to store it on Ethereum. Instead IPFS can be used for reliable and fast access at a very large scale and we can embed the generated CID, from the upload to IPFS, in our smart contract instead.
  \x
  Game\newline Assets\newline\reqref{F-C2}
  & \small Variable\footnote{Some games may include many promotional materials, whilst some could include none. Therefore, it is hard to estimate the expected size.} 
  & IPFS
  & \small This will represent any promotional material provided for the game that can be viewed on the game's store page. This will typically include cover art and a markdown file for the description and isn't required to purchase or download the game. The user will download this when they first view the game in the store.

  \vspace{1mm}
  \small For similar reasons as the hash tree, this data will be also be stored on IPFS and have its CID stored on Ethereum instead.
  \x
  Game Data
  & \textit{avg. 44GB~\footnote{Calculated based off of the top 30 games from SteamDB~\cite{noauthor_steam_nodate}.}}
  & Peers
  & This will the data required to run the game and will be fetched based upon the contents of the game's hash tree.

  \vspace{1mm}
  \small This data is very large and has restricted access so wouldn't be appropriate to store on either Ethereum or IPFS\footnote{IPFS and similar platforms provide no access control for the data stored there and any encryption based technique would be unviable.}. Therefore, this project will use a custom P2P network for sharing data, which is described in Section~\ref{subsec:design-p2p}.
  \\\bottomrule\bottomrule
  \caption{The different types of data required for each game.}
  \label{tab:data}
\end{longtable}

\noindent 
Swarm~\cite{hartman_swarm_1999} was considered as a decentralised storage and distribution platform over IPFS but was decided against as it would couple this project more tightly with Ethereum and the fact that IPFS has a much greater adoption.