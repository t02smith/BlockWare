
\subsection{Blockchain}\label{subsec:design-con-eth}

\subsubsection*{Type of Blockchain}

To satisfy \reqref{NF-M1} and \reqref{NF-M2}, we will need to use a public blockchain. This will benefit my project by:
\vspace{2mm}
\begin{itemize}
  \item being accessible to a larger user-base, which should boost both availability and scalability \reqref{NF-S1},
  \item reducing the risk of censorship \reqref{NF-M1}, and
  \item providing greater data integrity \reqref{NF-M4}
\end{itemize}

\vspace{2mm}\noindent Ethereum is a public blockchain that allows developers to publish their own distributed applications to it. It comes with an extensive development toolchain so is an obvious choice for this project \reqref{F-M4}.

\subsubsection*{Uploading Games}
\label{subsubsec:eth-data}

To satisfy \reqref{F-M1} and \reqref{F-M2}, the data stored on the blockchain will be used for the identification of games and will consist of the following fields, where \textit{italic} fields will be automatically-generated for the user when executing the upload function:

\begin{longtable}{ p{.2\textwidth} p{.75\textwidth} }
  \toprule
  \textbf{Name} & \textbf{Description}
  \\\midrule\midrule
  \multicolumn{2}{c}{\cellcolor{red!70}\textit{For each game}} 
  \\\midrule
  title & The name of the game.\\
  version & The version number of the game.\\
  \textit{release date} & The timestamp for when the game was uploaded.\\
  developer & The name of the developer releasing the game \reqref{NF-M3}.\\
  \textit{uploader} & The Ethereum address of the developer \reqref{NF-M3}.\\
  \textit{root hash} & The root hash of the game that uniquely identifies the game and is based upon its contents.\\
  previous version & The root hash of the most previous version of the game if it exists.\\
  price & The price of the game in Wei\\
  \textit{hash tree CID} & Required for downloading the hash tree folder from IPFS.\\
  \textit{assets CID} & Required for downloading the assets folder from IPFS.\\\midrule
  \multicolumn{2}{c}{\cellcolor{green}\textit{Managing the Collection of Games}} 
  \\\midrule
  \textit{library} & A mapping for storing all games uploaded to the network, where a game's root hash is the key used to find its information.\\
  \textit{game hashes} & Solidity doesn't allow us to enumerate maps so we will also store a list of hashes for all games uploaded.\\
  \textit{purchased} & A mapping which allows us to easily check if a user has purchased a game \reqref{F-M6}.
  \\\bottomrule\bottomrule
  \caption{All the data to be stored on the Ethereum blockchain}
\end{longtable}

\subsubsection*{Purchasing Content}

Users will purchase games from developers over Ethereum by transferring Ether \reqref{F-M5}. The user's address will then be added to the public record, on the smart contract, of all users who have purchased the game \reqref{F-M6}. Upon purchasing a game, a user will broadcast their new library to all of their peers.