
\subsection{Blockchain}\label{subsec:design-con-eth}

This section will describe how blockchain technology will enable the storage of game metadata and the purchasing of content using a distributed immutable record that can be trusted by any user. 
\x
To satisfy \reqref{NF-M1} and \reqref{NF-M2}, we will need to use a public blockchain. This will benefit this project by:
\vspace{2mm}
\begin{itemize}
  \item being accessible to more users, which will boost both availability and scalability \reqref{NF-S1},
  \item reducing the risk of censorship \reqref{NF-M1}, and
  \item providing greater data integrity \reqref{NF-M4}.
\end{itemize}

\newparagraph Ethereum is a public blockchain that allows developers to publish their own distributed applications to it. It comes with an extensive development toolchain so is an obvious choice for this project \reqref{F-M4}.

\subsubsection*{Uploading Games}
\label{subsubsec:eth-data}

To satisfy \reqref{F-M1} and \reqref{F-M2}, the data stored on the blockchain will be used for the identification of games. Table~\ref{tab:eth-data} shows the fields that will stored as part of the smart contract for each game and to manage the whole collection of games. Fields in \textit{italics} are generated for the user and non-italic fields are entered manually.

\begin{longtable}{ p{.2\textwidth} p{.75\textwidth} }
  \toprule
  \textbf{Name} & \textbf{Description}
  \\\midrule\midrule
  \multicolumn{2}{c}{\textit{Metadata for each game}} 
  \\\midrule\midrule
  title & The name of the game.\\
  version & The version number of the game.\\
  \textit{release date} & The timestamp for when the game was uploaded.\\
  developer & The name of the developer uploading the game \reqref{NF-M3}.\\
  \textit{uploader} & The Ethereum address of the developer \reqref{NF-M3}.\\
  \textit{root hash} & A unique fingerprint that identifies the game.\\
  previous version & The root hash of the most previous version of the game if it exists.\\
  \textit{next version} & The root hash of next update to this game if it exists. \\
  price & The price of the game in Wei.\\
  \textit{hash tree CID} & Required for downloading the hash tree from IPFS.\\
  \textit{assets CID} & Required for downloading the assets folder from IPFS.
  \\\midrule\midrule
  \multicolumn{2}{c}{\textit{Managing the Collection of Games}} 
  \\\midrule\midrule
  \textit{library} & A mapping for all games uploaded to the network, where a game's root hash is the key used to find its metadata.\\
  \textit{game hashes} & Solidity doesn't allow us to enumerate maps so we will also store a list of hashes for all games uploaded.\\
  \textit{purchased} & A mapping which allows us to easily check if a user has purchased a game \reqref{F-M6}.
  \\\bottomrule\bottomrule
  \caption{The data to be stored on Ethereum using a smart contract}
  \label{tab:eth-data}
\end{longtable}


\subsubsection*{Purchasing Content}

Users will purchase games from developers over Ethereum by transferring Ether \reqref{F-M5}. The user's address will then be added to a public record of all users who have purchased the game \reqref{F-M6}.
