\section{Project Organisation}

\subsection*{Agile Development}

The Agile Methodology~\cite{ilieva_analyses_2004} is a continuous development cycle that allows for developers to react to change in terms of the scope or requirements for a project and was used throughout this project to organise my time.


\subsubsection*{Sprints}
One useful advantage of dividing requirements into sprints is that it gives you a greater understanding of how requirements should be prioritised in terms of producing a minimally viable product.
\x
Sprints also gave me fixed time windows in which to complete and write tests for certain parts of the code.... 

\subsubsection*{Test Driven Development}
Test Driven Development~\cite{beck_test-driven_2003} TDD worked extremely well with Agile development as it meant that as I was completing requirements I was also testing their functionality. This meant that, in the long run, it was much easier to locate bugs and ensure the functionality of existing code through regression tests. 
\x
Tests were also useful in helping with me reason with my code. For example, if a new feature broke a lot of tests then the code might be too tightly coupled and should be considered for a refactor.
\x
One disadvantage is that writing tests are time consuming and, when working with distributed systems and or concurrent code, can be quite difficult to make. This made writing my application tedious.

