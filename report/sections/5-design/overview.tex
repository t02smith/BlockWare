\newpage
\section{Design Considerations}

% chktex-file 24
% chktex-file 8

\subsection{Data Types}
\label{subsec:design-data}

Table~\ref{tab:data} discusses the different types of data we are going to need to store and where they should be stored based upon their properties.

\begin{longtable}{ p{.12\textwidth} p{.1\textwidth} p{.11\textwidth} p{.58\textwidth} }
  \toprule
  \textbf{Data} & \textbf{Size} & \textbf{Location} & \textbf{Explanation}\\
  \midrule\midrule
  Game Metadata\newline\reqref{F-M1}
  & \small100 -- \newline200B
  & Blockchain
  & \small The minimal set of information required for the unique identification of each game. See Section~\ref{subsubsec:eth-data}.

  \vspace{1mm}
  \small This data is appropriate to store on the blockchain as it is public, small in size, and essential to the correct functioning of the application as all users will need to be able to discover all games. 
  \x
  Game Hash Tree\newline\reqref{F-M12}
  & \~ \small15KB
  & IPFS
  & \small The hash tree that will allow users to identify and verify blocks of data they need to download for a game. The user will download this immediately after purchasing the game.

  \vspace{1mm}
  \small This data is public but its size makes it costly to store on the blockchain at a large scale. IPFS will be used for fast, reliable access at a large scale and we can store the generated content-identifier CID in the blockchain instead.
  \x
  Game\newline Assets\newline\reqref{F-C2}
  & \small Variable\footnote{Some games may include many promotional materials, whilst some could include none. Therefore, it is hard to estimate the expected size.} 
  & IPFS
  & \small Any promotional material provided for the game that can be viewed on the game's store page. This should include cover art and a description file but isn't required to purchase or download the game. The user will download this when they first view the game in the store.

  \vspace{1mm}
  \small For similar reasons as the hash tree, this data will be also be stored on IPFS and have its CID stored on the blockchain instead.
  \x
  Game Data
  & \textit{avg. 44GB\footnote{Calculated based off of the top 30 games from SteamDB~\cite{noauthor_steam_nodate}.}}
  & Peers
  & The data required to run the game that is fetched based upon the contents of the game's hash tree.

  \vspace{1mm}
  \small This data is very large and has restricted access so wouldn't be appropriate to store on either the blockchain or IPFS. Therefore, this project will use a custom P2P network for sharing data, which is described in Section~\ref{subsec:design-p2p}.
  \\\bottomrule\bottomrule
  \caption{The different types of data required for each game.}
  \label{tab:data}
\end{longtable}
% \subsection*{Architecture}

% Figure~\ref{fig:architecture-diagram} shows the architecture of this application:

% \begin{figure}[ht]
%   \centering
%   \includegraphics*[width=\textwidth]{assets/images/diagrams/architecture-diagram.png}
%   \caption{Architecture of the application}
%   \label{fig:architecture-diagram}
% \end{figure}

% \subsection*{Sequence Diagram}

% Figure~\ref{fig:sequence-diagram}, shows the main interactions between actors in the application.

% \begin{figure}[ht]
%   \centering
%   \includegraphics[width=.95\textwidth]{assets/images/diagrams/seqeunce-diagram.png}
%   \caption{A sequence diagram showing some of the main interactions within this application}
%   \label{fig:sequence-diagram}
% \end{figure}



\subsection{Blockchain}

\subsubsection*{Type of Blockchain}

To satisfy \reqref{NF-M1} and \reqref{NF-M2}, we will need to use a public blockchain, which will benefit our project by:
\vspace{2mm}
\begin{itemize}
  \item being accessible to a larger user-base, which should boost availability and scalability \reqref{NF-S1},
  \item reducing the risk of censorship \reqref{NF-M1}, and
  \item providing greater data integrity \reqref{NF-M4}
\end{itemize}

\vspace{2mm}\noindent Ethereum is a public blockchain that allows developers to publish their own distributed applications to it. It comes with an extensive development toolchain so is an obvious choice for this project \reqref{F-M8}.

\subsubsection*{Uploading Games}
\label{subsubsec:eth-data}

To satisfy \reqref{F-M1}, the data stored on the blockchain will be used for the identification games and will consist of the following fields, where \textit{italic} fields will be automatically-generated for the user when executing the upload function:

\begin{longtable}{ p{.2\textwidth} p{.75\textwidth} }
  \toprule
  \textbf{Name} & \textbf{Description}
  \\\midrule\midrule
  \multicolumn{2}{c}{\cellcolor{red!70}\textit{For each game}} 
  \\\midrule
  title & The name of the game.\\
  version & A version number of the game.\\
  release date & When the game was released.\\
  developer & The name of the developer releasing the game.\\
  previousVersion & The root hash of the most previous version of the game if it exists.\\
  price & The price of the game in Wei\\
  \textit{uploader} & The Ethereum address of the developer.\\
  \textit{root hash} & The root hash of the game that uniquely identifies it and is based upon its contents.\\
  \textit{Hash Tree CID} & Required for downloading the Hash Tree folder of IPFS.\\
  \textit{Assets CID} & Required for downloading the assets folder of IPFS.\\\midrule
  \multicolumn{2}{c}{\cellcolor{green}\textit{Other}} 
  \\\midrule
  \textit{library} & A mapping for storing all games uploaded to the network, where a game's root hash is the key used to find its information.\\
  \textit{gameHashes} & Solidity doesn't allow us to enumerate maps so we will also store a list of hashes for all games uploaded.\\
  \textit{purchased} & A mapping which allows us to easily check if a user has purchased a game.
  \\\bottomrule\bottomrule
  \caption{All the data to be stored on the Ethereum blockchain}
\end{longtable}

\subsubsection*{Purchasing Content}

Users will purchase content from developers over Ethereum using Ether \reqref{F-M9} and this will be recorded on the blockchain \reqref{F-M10}. Any user can see which other users have purchased the game users can prove this between each other using their public/private keys.
\subsection{Distributed File Sharing}
\label{subsec:design-p2p}

\subsubsection*{Hash Tree}
\label{subsubsec:hash-tree}

The hash tree of a given directory is a tree object that stores information about its structure and contents. This is used to tell users what information they need to download, where it goes and what its contents should be. For every file, the hash tree stores a series of SHA-256 hashes. 

\begin{figure}[ht]
  \centering
  \includegraphics[width=.85\textwidth]{assets/images/diagrams/block-body.png}
  \caption{The structure of a hash tree}
  \label{fig:hash-storage}
\end{figure}

\subsubsection*{Uploading Content}
\label{subsubsec:upload-content}

For a developer to upload their game \reqref{F-M5} they must provide the required metadata outlined in Section~\ref{subsubsec:eth-data} as well as the location of the game in storage. A hash tree is then produced of the game and this is uploaded to IPFS, and the metadata about the game is uploaded to Ethereum.

\subsubsection*{Downloading Content}

Like mentioned in Section~\ref{subsec:design-data}, it is impractical to store the game's data on the blockchain or even IPFS. Instead we will consider ideas from decentralised file-sharing networks, like BitTorrent, to facilitate the distribution of content.
\x
Games are content addressable and are identified by their root hashes, which are stored on the blockchain and are calculated from the hash tree. Users will send messages using this hash to identify other users who are also interested in the same content \reqref{F-M3}, so they can share data. When two nodes connect to share content the node seeking content will:

\begin{enumerate}
  \item send their ethereum address along with an encrypted message to prove their address. The address is then looked up on the purchase list stored in the game's entry on the blockchain \reqref{F-S2}.
  \item Request individual shards from the node using the shard's hash \reqref{F-M2}.
  \item Use the hash tree, which has been fetched from IFPS, to verify the shard's contents \reqref{F-M7}.
  \item Send an encrypted encrypted certificate that the sender can use to prove their contribution.
  \item Repeat this for an arbitrary number of shards.
\end{enumerate}

\subsubsection*{Updating Content}

To satisfy \reqref{F-M4}, developers will perform the same steps outlined in Section~\ref{subsubsec:upload-content} but must also provide the root hash of the most previous version of the game. Any users who have purchased the previous version, will be added to the list of users who have purchased the new version. Additionally, this will include the restriction that only the original uploader can upload an update for their game \reqref{NF-S2}.
\x
Each version is considered as its own game and will require users to download the updated version separately. Whilst this isn't reflective of how updates are typically managed, this will be acceptable for the scope of this project and any changes will be considered as a future extension to this project.

\subsubsection*{Downloadable Content}
% TODO
\tbd

% {F\_S3}) for their games that users will purchase separately. Each DLC will need:

% \begin{enumerate}
%   \item \textbf{Dependency} A reference to the oldest version of the game they apply to, and
%   \item \textbf{Previous Version} A pointer to the previous version of the DLC.
% \end{enumerate}

% \begin{figure}[ht]
%   \centering
%   \includegraphics[width=.85\textwidth]{assets/images/diagrams/software.png}
%   \caption{How blocks relate to each other. An update will reference the previous version whilst a DLC will reference, which piece of software and version it is dependent on.}
% \end{figure}

\subsubsection*{Proving Contribution}

When a user successfully downloads a shard of data, they will reply with a confirmation certificate to prove that they have downloaded the game. Each Ethereum address is related to a public/private key pair so a confirmation certificate will be encrypted by a given node's private key to prove authenticity. A user will prove their contribution \reqref{F-S1} by sending a collection of certificates to the uploader, who will validate them and reward the user accordingly.

\subsection*{Downloading Content}

Games will be content addressable, using their root hash stored on the blockchain. This will be used to help connect nodes who are interested in the same content (\textbf{F\_M3}). Once a user connects to a node it will:
\vspace{2mm}
\begin{enumerate}
  \item Send their proof of purchase (\textbf{F\_S2}).
  \item request individual shards from the node using the shard's hash (\textbf{F\_M2}),
  \item use the metadata from the blockchain to verify the shard's contents (\textbf{F\_M7}),
  \item send a confirmation message that proves the successful transfer of a block (\textbf{F\_S1}),
  \item and repeat this until the entirety of the game is installed (\textbf{F\_M6}).
\end{enumerate}

\vspace{2mm}\noindent As the blockchain will only be used to store metadata about games uploaded to the network, not every node will have each game installed. Instead, this project will take ideas from networks like BitTorrent, where nodes will seek out peers who have the game installed using a tracker hosted by the uploader.

\subsection*{Updating Content}

To satisfy \textbf{F\_M4}, developers will perform the same steps as in \textbf{Uploading Content} but will also include the hash of a previous block that contains the older version of the game. This relationship will allow users ownership to all versions of a game they have purchased and not just a singular version. This will include the restriction that only the original uploader can upload an update for their game (\textbf{NF\_S2}).

\vspace{2mm}\noindent Each version will be considered as a standalone game and will require users to download the entirety by scratch. Whilst this isn't reflective of how updates are typically managed, this will be acceptable for the scope of this project and any changes will be considered as a future extension to this project.

\subsubsection*{Downloadable Content}

Developers will typically offer paid or free DLC (\textbf{F\_S3}) for their games that users will purchase separately. Each DLC will need:

\begin{enumerate}
  \item \textbf{Dependency} A reference to the oldest version of the game they apply to, and
  \item \textbf{Previous Version} A pointer to the previous version of the DLC.
\end{enumerate}

\begin{figure}[ht]
  \centering
  \includegraphics[width=.85\textwidth]{assets/images/diagrams/software.png}
  \caption{How blocks relate to each other. An update will reference the previous version whilst a DLC will reference, which piece of software and version it is dependent on.}
\end{figure}

\subsection*{Proving Contribution}

A purchase certificate will include a confidential \textit{seeder token}. When a user successful downloads a shard of data they will send a confirmation message, including their seeder token, that is encrypted with the game developer's public key. A user will prove their contribution (\textbf{F\_S1}) by sending a collection of these messages to the developer, who can decrypt and validate the tokens.
