\section{Sections}

\subsection{Distributed File Sharing}

Users running this application will be part of a distributed network of peers that are connected using TCP connections and will send structured messages to each other. Section~\ref{subsubsec:commands} describes the commands that peers will send to each other.
\x
Go \url{https://go.dev/} was chosen to write this in due its simple syntax, strong built in library, and excellent support for concurrent programming. On top of this, Go has good third party libraries that will be useful for interacting with Ethereum.

\subsubsection*{Commands}
\label{subsubsec:commands}

These commands typically consist of request/response pairs where a peer will request a piece of information and be sent it. Peers will not wait for a response after sending a request and will expect to receive it at a later point in time. This means that a peer can still respond to a given peers requests before receiving a response from it.

\begin{longtable}{p{.4\textwidth} p{.57\textwidth}}
  \toprule
  \textbf{Message Format} & \textbf{Description}\\
  \midrule\midrule
  LIBRARY
  & Request that a peer sends their library of games in the form of a BLOCK message.\\
  GAMES;$[hash_1]$;$[hash_2]$;\ldots;
  & The user sends a list of their games as a series of unique root hashes. These root hashes will map to games on the blockchain.\\
  \midrule
  BLOCK;$[gameHash]$;$[blockHash]$;
  & The user will request a block of data off of a user by sending the root hash of the game and the hash of the block being requested. The response will be a SEND\_BLOCK message.\\
  SEND\_BLOCK;$[gameHash]$;\newline $[blockHash]$;$[data]$;
  & The user sends a block of data in response to a BLOCK message.\\
  \midrule
  ERROR;$[message]$
  & An error message that can be used to prompt a peer to resend a message.\\
  \bottomrule\bottomrule
\end{longtable}

\subsubsection*{Downloading}

A downloader process will be used for each game being downloaded, where they will select blocks based upon:

\begin{itemize}
  \item What file they are from, where downloading blocks from the same file is preferred.
  \item How many users have a given block, where rarer blocks are a priority to boost availability.
\end{itemize}

\vspace{2mm}\noindent
Each request is sent over a channel for the peer to read and forward the request to the relevant neighbouring peers.

\subsection{User Interface}

This application will have a GUI where users can interact with the platform and will need to include the following pages:

\begin{itemize}
  \item \textbf{Library} The user's collection of owned games, where they can view details of each game owned as well as manage their download status.
  \item \textbf{Store} Where user's can find new games that have been uploaded by other users and purchase them.
  \item \textbf{Upload} Where user's can fill in details about a new game and have it be processed and uploaded to the blockchain.
  \item \textbf{Downloads} Where user's can track all of their owngoing downloads.
  \item \textbf{Peers} Where users can manage their list of connected peers.
\end{itemize}

\vspace{2mm}\noindent
Wails \url{https://wails.io/} was chosen to create the GUI because it allows us to use a reactive web-based framework and have it interface directly with controller functions written in Go. This meant building the UI was straight-forward and allowed me to use tools that I was familiar with. These were Vue.js v3 \url{https://vuejs.org/}, \url{https://pinia.vuejs.org/}, and SCSS \url{https://sass-lang.com/}.

\subsection*{Ethereum}

The smart contract will be written in Solidity \url{https://docs.soliditylang.org/en/v0.8.19/}, and will have to be able to store the data mentioned in Section~\ref{subsubsec:eth-data} as well as perform the following functions:

\begin{itemize}
  \item \textbf{uploadGame} Upload a new game to the blockchain provided it doesn't already exist. Once uploaded, emit an event to let all peers aware of the new game.
  \item \textbf{purchaseGame} A peer purchases a game they don't already own and transfer ETH to the original uploader and are added to the list of authorised users.
\end{itemize}

\vspace{2mm}\noindent
The go-ethereum package \url{https://geth.ethereum.org/} will allow us to interact with the ethereum blockchain and Abigen \url{https://docs.avax.network/specs/abigen} will allow us to compile any smart contracts to Go code. This will allow us to interact with our smart contract on ethereum using a set of Go functions.