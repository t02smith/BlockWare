
\section{P2P File Sharing}
\label{sec:lit-p2p}

As users will only have the data they are interested in, it is important for me to look at implementations of distributed file-sharing systems that connects users based on the content they are interested in.
\x
Table~\ref{tab:lit-review-p2p} looks at various implementations of P2P file-sharing networks.

\small
\begin{longtable}{ p{0.15\textwidth} p{0.8\textwidth} }
  \toprule
  \textbf{System} & \textbf{Description of Solution}
  \\\midrule\midrule
  IPFS~\cite{benet_ipfs_2014}
  & IPFS is a set of protocols for transferring and organising data over a content-addressable, peer-to-peer network. IPFS is open source and has many different implementations, such as Estuary~\cite{noauthor_estuary_nodate-1} or Kubo~\cite{noauthor_ipfskubo_2023}.
  %
  \x
  Swarm~\cite{hartman_swarm_1999}
  & Swarm is a distributed storage solution linked with Ethereum that has many similarities with IPFS~\cite{pouwelse_bittorrent_2005}. It uses an incentive mechanism, Swap (Swarm Accounting Protocol), that keeps track of data sent and received by each node in the network and then the payment owed for their contribution.
  %
  \x
  BitTorrent~\cite{pouwelse_bittorrent_2005}
  & See Section~\ref{sec:bittorrent}.
  %
  \x
  AFS~\cite{morris_andrew_1986,howard_scale_1988}
  & The Andrew File System was a prototype distributed system by IBM and Carnegie-Mellon University in the 1980s that allowed users to access their files from any computer in the network.
  %
  \x
  Napster~\cite{saroiu_measurement_2001}
  & In Napster, a central cluster of servers will maintain an index of every file in the network, as well as which peers have that file. Clients will query the cluster for this information and will choose peers based upon their bandwidth.
  %
  \x
  Gnutella~\cite{saroiu_measurement_2001}
  & In Gnutella, nodes form an overlay network and will discover other peers through \textit{ping-pong} messages, where any node that receives a \textit{ping} message will forward it to their neighbours and send a \textit{pong} message to the originator. 
  A user will flood a download request to their peers until they find a suitable peer to download off of.
  \\\bottomrule\bottomrule
  \caption{Various distributed file systems.}
  \label{tab:lit-review-p2p}
\end{longtable}
\normalsize

\noindent
Some of the common themes found include:

\begin{itemize}
  \item \textbf{Trust} Nodes are typically anonymous so users have to trust that the data they download isn't malicious.
  \item \textbf{Illegal Content} There are no measures in place to stop or prevent the distribution of illegal content across these networks.
  \item \textbf{Payment} None of these networks natively support payment and thus don't support the access control systems required for payment systems.
\end{itemize}

\newparagraph
It is clear that using blockchain in conjunction with these systems could mitigate a lot of their issues and thus a combination of these two ideas will be used for this project.