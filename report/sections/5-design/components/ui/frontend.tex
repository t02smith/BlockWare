\subsection{Frontend \& Controller}

\subsubsection{Frontend}\label{subsubsec:frontend}

This application will have a GUI \reqref{F-S2} \reqref{NF-S2} where users can interact with the platform. Having a GUI is essential to making the platform as easy to use as possible so that it is accessible to new users. At minimum it will need to include the following pages:

\begin{itemize}
  \item \textbf{Library} The user's collection of owned games, where they can view details for each game as well as manage their download status. Users should be able to view both old and new versions of a game and check for any new updates.
  \item \textbf{Store} Where user's can find and purchase new games that have been uploaded by other users. Games should be searchable by their root hash.
  \item \textbf{Upload} Where users can fill in details about their new or updated game and have it be uploaded to the network.
  \item \textbf{Downloads} Where user's can track all of their ongoing downloads and see their progress.
  \item \textbf{Peers} Where users can manage their list of connected peers. Here a user can form new connections, break existing ones and request contribution data from their peers.
  \item \textbf{Help} A help page to describe the application and all of its functionality \reqref{NF-C1}
\end{itemize}

\newparagraph
To satisfy \reqref{NF-M3}, a developer must always be displayed with both their chosen name and their Ethereum address. A developer should publically provide their Ethereum address to ensure their users can identify it.

\subsubsection{Controller}

The Controller will be represented as a set of interface functions that allow the backend and frontend code to communicate. This can be done to trigger actions such as starting a game download or to fetch data like the list of a user's owned games.
