
\section{P2P File Sharing}
\label{sec:lit-p2p}

It is unreasonable to expect every node to have a copy of each game uploaded to the blockchain so data will be fragmented across the network. This project will use ideas from various P2P file-sharing networks to help connect nodes interested in the same content Table~\ref{tab:lit-review-p2p} shows some example p2p file-sharing networks.
\x
The main issues involving these networks are:

\begin{enumerate}
  \item \textbf{Trust} Nodes are typically anonymous and you can never fully trust that what you're downloading isn't malicious, and
  \item \textbf{Payment} These platform don't allow users to pay for content and are generally large sources of piracy. 
\end{enumerate}

\begin{longtable}{ p{0.15\textwidth} p{0.8\textwidth} }
  \toprule
  \textbf{System} & \textbf{Description of Solution}
  \\\midrule\midrule
  IPFS~\cite{benet_ipfs_2014}
  & IPFS is a content-addressable, block storage system which forms a Merkle DAG, a data structure that allows the construction of versioned file systems, blockchains and a Permanent Web.
  %
  \x
  BitTorrent~\cite{pouwelse_bittorrent_2005}
  & BitTorrent is a p2p file-sharing system that has user bartering for chunks of data in a tit-for-tat fashion, which provides incentive for users to contribute to the network. More on BitTorrent can be found in Section~\ref{sec:bittorrent}
  %
  \x
  AFS~\cite{morris_andrew_1986,howard_scale_1988}
  & The Andrew File System was a prototype distributed system by IBM and Carnegie-Mellon University in the 1980s that allowed users to access their files from any computer in the network.
  %
  \x
  Napster~\cite{saroiu_measurement_2001}
  & Napster uses a cluster of centralized servers to maintain an index of every file currently available and which peers have access to it. A node will maintain a connection to this central server and will query it to find files; the server responds with a list of peers and their bandwidth and the node will form a connection with one or many of them and download the data.
  %
  \x
  Gnutella~\cite{saroiu_measurement_2001}
  & Gnutella nodes form an overlay network by sending \textit{ping-pong} messages. When a node sends a \textit{ping} message to their peers, each of them replies with a \textit{pong} message and the \textit{ping} is forwarded to their peers. To download a file, a node will flood a message to its neighbors, who will check if they have and return a message saying so; regardless, the node will continue to flood their request till they find a suitable node to download off of.
  \\\bottomrule
  \caption{Various global distributed file systems.}
  \label{tab:lit-review-p2p}
\end{longtable}