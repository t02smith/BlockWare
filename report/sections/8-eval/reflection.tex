\section{Reflection}

\subsection{Risk Assessment}

Below I'll detail some the risks mentioned in Section~\ref{sec:risk-assessment} that came up during development and how my mitigation strategy helped me cope with the issues as they came up.

\paragraph*{Difficulty with blockchain development}
As expected there were several difficulties with blockchain development:

\begin{itemize}
  \item A decent amount of the documentation is minimal or outdated. As blockchain development is a smaller field there are little in the ways of forum postings.
  \item Some of the libraries used had bugs in them that hindered development. For example, I could deploy the smart contract to the Sepolia testnet from Remix easily but not through Geth. 
\end{itemize}

\vspace{2mm}\noindent
Despite the problems mentioned above, I was able to complete the blockchain aspect of the project successfully. Effective use of sprint planning meant that I had considered this to be a potential issue and had allocated time to being able to deal with it.

\paragraph*{The application is not finished}
Section~\ref{sec:sprints} shows that not all of the requirements were finished and Section~\ref{sec:design-lim} discusses the limitations with my current implementation. It was expected that this project would not be entirely finished but a lot of the major requirements were met.
\x
The application ended up being incredibly large and took a lot more time to implement and test than expected. Table~\ref{tab:cloc} shows a breakdown of the source code. Agile development could only help me so much so by having a hard cut-off point for the implementation I was still left with plenty of time to complete the report.

\begin{longtable}{ | r | r | r | r | }
  \hline
  \textbf{Language} & \textbf{Files} & \textbf{Comments} & \textbf{Code}
  \\\hline
  Go
  & 45
  & 829
  & 5006
  \\\hline
  Go Tests
  & 26
  & 665
  & 2836
  \\\hline
  Vue.js Components
  & 17
  & 141
  & 2414
  \\\hline
  JavaScript
  & 19
  & 88
  & 377
  \\\hline
  Solidity
  & 1
  & 33
  & 46
  \\\hline
  \caption{The lines of code written for this project calculated using CLOC \url{https://github.com/AlDanial/cloc}}
  \label{tab:cloc}
\end{longtable}

\subsection{Agile Development}

The use of an agile methodology alongside test-driven development meant I could incrementally expand the scope of my application and ensure that existing code was tested sufficiently. Separating my implementation into three sprints benefitted me by:

\begin{itemize}
  \item having a smaller set of requirements to focus on at once helped me to feel less overwhelmed,
  \item working on the most important aspects first to ensure I was able to produce a minimum viable product, and
  \item taking time in-between sprints to take a break from the project and prepare for the next sprint.
\end{itemize}



\subsection{What Went Well}



\subsection{What Could Have Gone Better}

\paragraph*{Complexity}
Overall, I felt that this project felt too large and I had to commit an extremely large amount of time to complete the application and its tests. This resulted in a massive codebase.



\paragraph*{Large Scale Testing}
Like anticipated in Section~\ref{sec:risk-assessment}, this application was hard to test at a large enough scale to emulate real world usage.
