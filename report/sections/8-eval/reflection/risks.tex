\subsection*{Risk Assessment}

% \paragraph*{Difficult with blockchain development}

% Some of the difficulties I found with blockchain development were:

% \begin{itemize}
%   \item 
% \end{itemize}

\paragraph*{The application is not finished}
Section~\ref{tab:sprint-3} shows the scoped requirements that were not met and Section~\ref{sec:design-lim} shows some of the limitations within my design when compared to similar platforms. However, use of MoSCoW prioritisation and sprint planning meant I was still able to produce a largely functional application that met the majority of the requirements and having a cut off point for implementation ensured I had sufficient time to complete this report.
\x
One reason for not finishing was the size of the project, which took a considerable amount of time to implement and test. Table~\ref{tab:cloc} shows a breakdown of the source code.

\begin{longtable}{ r r r r }
  \toprule
  \textbf{Language} & \textbf{Files} & \textbf{Comments} & \textbf{Code}
  \\\midrule\midrule
  Go
  & 44
  & 683
  & 4621
  \\
  Go Tests
  & 29
  & 791
  & 3205
  \\
  Vue.js Components
  & 18
  & 147
  & 2855
  \\
  JavaScript
  & 19
  & 74
  & 209
  \\
  Solidity
  & 1
  & 30
  & 51
  \\\bottomrule\bottomrule
  \caption{The lines of code written for this project calculated using CLOC~\cite{noauthor_aldanialcloc_nodate} excluding any auto-generated code.}
  \label{tab:cloc}
\end{longtable}

\paragraph*{Lack of large-scale testing infrastructure}
Testing distributed applications is challenging due to many factors, such as having to source homogenous devices, having access to networks that would allow me host a peer, and more. However, several useful results were produced from the benchmarks (Section~\ref{sec:benchmark}) and acceptance tests (Section~\ref{sec:acc-tests}) that can be used to improve the application. If this project were to move forward then testing it on a large-scale would be essential.