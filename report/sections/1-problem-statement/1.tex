
\chapter{Problem Statement}

\section{The Problem}
\label{sec:problem}

Video games are often large and highly popular pieces of software that are typically distributed for developers by a third party platform like Steam or Epic Games. Whilst these platforms provide benefits such as availability, and some social features they have some major downsides that include:
\vspace{1mm}
\begin{enumerate}[label=(\alph*)]
  \item taking a large cut of all revenue, \newline\textit{Steam take a 30\% cut~\cite{marks_report_2019,brown_valve_2021}}
  \item being vulnerable to censorship from governments, \newline\textit{The Chinese version of Steam is heavily censored~\cite{steamdb_steam_2021}}
  \item the user's access to their games is linked to the platform. \newline\textit{If the platform shuts down, the user loses all their games}
\end{enumerate}
\vspace{1mm}

\section{Goals}

The goal of this project is to implement a large-scale distribution platform that will allow game developers to release and continuously update their games on a public network by directly interacting with their users. This is in the aim to provide greater profits to developer's, freedom from censorship, and better digital ownership for the user.

\section{Scope}

This project will consist of the following components:

\begin{enumerate}
  \item An Ethereum smart contract that will allow us to maintain a library of games that can be queried and added to by any user.
  \item A local application to be run by users to interact with the smart contract and allow users to join a peer-to-peer network where they can download and upload games to.
\end{enumerate}

\vspace{2mm}\noindent
Both sections will need to pass a series of acceptance tests to ensure that they meet the requirements I set out in Section~\ref{subsec:requirements}. Further tests will consider how the application fares in a live environment as well as a local development one. 
