
\section{Benefits}

This application presents the following benefits when compared with centralised software marketplaces:

\begin{itemize}
  \item \textbf{Decentralised} The use of decentralised platforms like Ethereum mean that no one party can control the platform. This gives much greater freedom over what can be uploaded and removes reliance on a single entity to maintain.
  \item \textbf{Direct Interaction} Users and developers will interact directly instead of via a middle-man, which removes the recurring cost of one for developers.
  \item \textbf{}
\end{itemize}

\section{Limitations}\label{sec:design-lim}

This application presents the following limitations when compared with a centralised software marketplace:

\begin{itemize}
  \item \textbf{No Social Features} Social features, such as friends or achievements, were not included within the scope of this project.
  \item \textbf{Availability} Section~\ref{subsec:availability} highlights the issue of availability within P2P file-sharing systems and it is likely this platform will face similar issues.
  The use of a contribution system was implemented to help identify those users who have been contributing but there is no automatic rewards system\footnote{An example would be how the micro-payment system works in Swarm~\cite{hartman_swarm_1999}}.
  \item \textbf{Inefficient Updates} As updates are treated as individual games, they will require users to download the entire game again. This is highly inefficient and results in lots of duplicate data being downloaded.
  \item \textbf{Hash Trees} Modelling data as a hash tree means that each file will need at least one block so a game may have a large amount of blocks for not a lot of data and as each block has to be requested separately this will add a lot network overhead. 
\end{itemize}