\section{Integration Testing}

\subsection*{Profiles}

A profile is designed to mimic a specific type of peer to allow me to test my application under more realistic conditions that didn't exclusively involve ideal peers. Table~\ref{tab:profiles} describes each profile, their purpose and the outcomes from using them.

\small
\begin{longtable}{p{.12\textwidth} p{.39\textwidth} p{.39\textwidth} }
  \toprule
  \textbf{Name} & \textbf{Purpose} & \textbf{Outcome}
  \\\midrule\midrule
  \textbf{Listen Only}
  & A peer who will listen and respond to all requests perfectly but will never request anything.
  & This implementation was used for benchmarking and development as a perfect peer.
  \\
  \textbf{Send Only}
  & This peer will never respond to requests but will periodically send them and is supposed to represent a selfish client.
  & This indicated the need for a \textit{reputation} score for each peer to determine whether their requests should be replied to.
  \\
  \textbf{Spammer}
  & This peer will flood its peers with the same message over and over again. This represents a potentially malicious client.
  & This indicated the need for ...
  \\
  \textbf{Unreliable}
  & This peer will pseudo-randomly not respond to messages or send incorrect data in response.
  & This also indicated the need for a \textit{reputation} score where users who are unreliable should be given lower priority.
  \\\bottomrule\bottomrule
  \caption{The different profiles used to simulate real-world peers.}
  \label{tab:profiles}
\end{longtable}
\normalsize

\noindent
The use of various profiles showed that this application needed some metric by which to rank peers to allow us to prioritise requests and choose who to send requests to. However, due to time constraints this was not completed but should be considered moving forward.

