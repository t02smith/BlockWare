\subsection*{Risk Evaluation}\label{subsec:risk-reflection}

Table~\ref{tab:risks-happened} looks at which risks occurred during the this project and how effective my mitigation stratgeies were.

\begin{longtable}{p{.2\textwidth} p{.73\textwidth}}
  \toprule
  \textbf{Risk} & \textbf{Explanation}
  \\\midrule\midrule
  The application is not finished
  & Several requirements were not met, as shown in Section~\ref{tab:sprint-3}, and this can be attributed to the project simply being too large for the time frame I had. 
  

  However, this risk was highly anticipated and by using sprints I was able to focus my attention to the most important requirements and ensure that the application I had was still successful in terms of the goals of this project.
  \x
  Lack of\newline large-scale testing\newline infrastructure
  & The lack of devices available to me made this application difficult to test in a \textit{real} environment but the mitigations suggested allowed me to prove that my application worked correctly in a test-network and would can scale effectively.
  \x
  Difficulty with\newline blockchain\newline development
  & Being new to this field meant I faced several setback when working with blockchain technology. However by dedicating time to learning and practicing before incorporating it into my project, I was able to successfully complete that section.
  \\\bottomrule\bottomrule
  \caption{The risks which occurred during this project.}
  \label{tab:risks-happened}
\end{longtable}

