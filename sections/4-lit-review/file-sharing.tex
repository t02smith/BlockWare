
\section{P2P File Sharing}

These applications involve a distributed network of computers that share data with each other without the need for a central party. Table~\ref{tab:lit-review-p2p} shows some example p2p file-sharing networks.
\x
One of the main issues with these networks come from their anonymity property in that you can never fully trust that what you're downloading isn't malicious. On top of this, these platforms don't offer a way for user to pay for the content they are downloading, which means that legal, proprietary content is rarely distributed over them.

\begin{longtable}{ p{0.15\textwidth} p{0.8\textwidth} }
  \toprule
  \textbf{System} & \textbf{Description of Solution}
  \\\midrule\midrule
  IPFS~\cite{benet_ipfs_2014}
  & IPFS is a content-addressable, block storage system which forms a Merkle DAG, a data structure that allows the construction of versioned file systems, blockchains and a Permanent Web.
  %
  \x
  BitTorrent~\cite{pouwelse_bittorrent_2005}
  & BitTorrent is a p2p file-sharing system that has user bartering for chunks of data in a tit-for-tat fashion, which provides incentive for users to contribute to the network. More on BitTorrent can be found in Section~\ref{sec:bittorrent}
  %
  \x
  AFS~\cite{morris_andrew_1986,howard_scale_1988}
  & The Andrew File System was a prototype distributed system by IBM and Carnegie-Mellon University in the 1980s that allowed users to access their files from any computer in the network.
  %
  \x
  Napster~\cite{saroiu_measurement_2001}
  & Napster uses a cluster of centralized servers to maintain an index of every file currently available and which peers have access to it. A node will maintain a connection to this central server and will query it to find files; the server responds with a list of peers and their bandwidth and the node will form a connection with one or many of them and download the data.
  %
  \x
  Gnutella~\cite{saroiu_measurement_2001}
  & Gnutella nodes form an overlay network by sending \textit{ping-pong} messages. When a node sends a \textit{ping} message to their peers, each of them replies with a \textit{pong} message and the \textit{ping} is forwarded to their peers. To download a file, a node will flood a message to its neighbors, who will check if they have and return a message saying so; regardless, the node will continue to flood their request till they find a suitable node to download off of.
  \\\bottomrule
  \caption{Various global distributed file systems.}
  \label{tab:lit-review-p2p}
\end{longtable}