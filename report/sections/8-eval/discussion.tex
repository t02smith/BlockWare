\section{Discussion}

Due to the scale and complexity of the platforms this project was aiming to replace, this project was never going to be more than a proof-of-concept as to what a distributed games marketplace could look like. 
However, this application does have several benefits over its centralised competitors, namely:

\begin{itemize}
  \item \textbf{Privacy} A user's personal information and usage isn't collected. Traditional platforms require users to enter personal information and will actively collect data about a user's actions through the platform.
  \item \textbf{Ownership} A user's ownership of a game isn't tied to a single platform and use of Ethereum means that a user's ownership is upheld by all computers in the network.
  \item \textbf{Censorship} Similar to the previous point, no one party has control over the platform so it is much harder for third parties, such as governments, to restrict the content uploaded to it.
  \item \textbf{Profits} Developers and gamers communicate directly and this means developer's won't have to pay a hefty fee for a middle-man. This will result in potentially larger profits for the developers.  
\end{itemize}

\vspace{2mm}\noindent
This project presents interesting ideas surrounding how games, and other proprietary software, can be distributed without the need for a middle-man. This follows modern web ideas of taking away reliance on large data-centres and having a system built and maintained by the people who use it. Moreover, this type of project is usually open source, which can help attract community contributions to improve the security and user experience of the application.
