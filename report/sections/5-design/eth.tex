
\subsection{Blockchain}

\subsubsection*{Type of Blockchain}

To satisfy \textbf{NF\_M1} and \textbf{NF\_M2}, we will need to use a public blockchain, which will benefit our project by:
\vspace{2mm}
\begin{itemize}
  \item being accessible to a larger user-base, which should boost availability and scalability (\textbf{NF\_S1}),
  \item reducing the risk of censorship (\textbf{NF\_M1}), and
  \item providing greater data integrity (\textbf{NF\_M4})
\end{itemize}

\vspace{2mm}\noindent Ethereum is a public blockchain that allows developers to publish their own distributed applications to it. It comes with an extensive development toolchain so is an obvious choice for this project (\textbf{F\_M8}).

\subsubsection*{Data to Store}
\label{subsubsec:eth-data}

Like mentioned in Section~\ref{subsec:design-data}, the data stored on the blockchain is used for the identification of each game and will consist of the following fields, where \textit{italic} fields will be automatically-generated:

\begin{longtable}{ p{.2\textwidth} p{.75\textwidth} }
  \toprule
  \textbf{Name} & \textbf{Description}
  \\\midrule\midrule
  title & The name of the game.\\
  version & A version number of the game.\\
  release date & When the game was released.\\
  developer & The name of the developer releasing the game.\\
  \textit{uploader} & The Ethereum address of the developer.\\
  \textit{root hash} & The root hash of the game that uniquely identifies it and is based upon its contents.\\
  \textit{IPFS address} & The ID of the hash tree of IPFS that can be used to download the hash tree before starting a download of a game.
  \\\bottomrule\bottomrule
\end{longtable}

\subsubsection*{Purchasing Content}

Users will purchase content from developers over Ethereum using Ether (\textbf{F\_M9}) and this will be recorded on the blockchain (\textbf{F\_M10}). Any user can see which other users have purchased the game users can prove this between each other using their public/private keys.