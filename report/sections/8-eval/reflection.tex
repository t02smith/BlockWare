\section{Reflection}

\subsection{Risk Assessment}

\paragraph*{Difficulty with blockchain development}
As expected there were several difficuluties with blockchain development:

\begin{itemize}
  \item A decent amount of the documentation is minimal or outdated. As blockchain development is a smaller field there are little in the ways of forum postings.
  \item Some of the libraries used had bugs in them that hindered development. For example, I could deploy the smart contract to the Sepolia testnet from Remix easily but not through Geth. 
\end{itemize}

\paragraph*{The application is not finished}
Section~\ref{} details the requirements that were not finished and Section~\ref{} discusses the limitations with my current implementation. It was expected that this project would not be entirely finished but a lot of the major requirements were met.
\x
The application ended up being incredibly large and took a lot more time to implement and test than expected. Table~\ref{tab:cloc} shows a breakdown of the source code.

\begin{longtable}{ | r | r | r | r | }
  \hline
  \textbf{Language} & \textbf{Files} & \textbf{Comments} & \textbf{Code}
  \\\hline
  Go
  & 45
  & 829
  & 5006
  \\\hline
  Go Tests
  & 26
  & 665
  & 2836
  \\\hline
  Vue.js Components
  & 17
  & 141
  & 2414
  \\\hline
  JavaScript
  & 19
  & 88
  & 377
  \\\hline
  Solidity
  & 1
  & 33
  & 46
  \\\hline
  \caption{The lines of code written for this project calculated using CLOC \url{https://github.com/AlDanial/cloc}}
  \label{tab:cloc}
\end{longtable}

\subsection{What Went Well}



\subsection{What Could Have Gone Better}

\paragraph*{Complexity}
Overall, I felt that this project felt too large and I had to commit an extremely large amount of time to complete the application and its tests. This resulted in a massive codebase.



\paragraph*{Large Scale Testing}
Like anticipated in Section~\ref{sec:risk-assessment}, this application was hard to test at a large enough scale to emulate real world usage.
