
\section{Blockchain-Based Cloud Storage}

Blockchain technology can be leveraged for distributed cloud storage to provide both public and private storage. In table~\ref{tab:blockhain cloud storage}, I detail some examples of how blockchain has been used to create cloud storage platforms:

\small
\begin{longtable}{ p{0.35\textwidth} p{0.6\textwidth} }
  \toprule
  \textbf{Paper} & \textbf{Description of Solution}
  \\\midrule\midrule
  Blockchain Based Data Integrity Verification in P2P Cloud Storage~\cite{yue_blockchain_2018}
  & This paper uses Merkle trees to help verify the integrity of data within a P2P blockchain cloud storage network. It also looks at how different structures of Merkle trees effect the performance of the system.
  %
  \x
  Deduplication with Blockchain for Secure Cloud Storage~\cite{li_deduplication_2018}
  & This paper describes a deduplication scheme that uses the blockchain to record storage information and distribute files to multiple servers. This is implemented as a set of smart contracts.
  %
  \x
  Block-secure: Blockchain based scheme for secure P2P cloud storage~\cite{li_block-secure_2018}
  & A distributed cloud system in which users divide their own data into encrypted chunks and upload those chunks randomly into the blockchain, P2P network. 
  %
  \x
  Blockchain-Based Medical Records Secure Storage and Medical Service Framework~\cite{chen_blockchain-based_2018}
  & Describes a secure and immutable storage scheme to manage personal medical records as well as a service framework to allow for the sharing of these records.
  %
  \x
  A Blockchain-Based Framework for Data Sharing With Fine-Grained Access Control in Decentralized Storage Systems~\cite{wang_blockchain-based_2018}
  & This solution uses IPFS, Ethereum and ABE technology to provide distributed cloud storage with an access rights management system using secret keys distributed by the data owner.
  %
  \x
  Blockchain based Proxy Re-Encryption Scheme for Secure IoT Data Sharing~\cite{manzoor_blockchain_2019}
  & An IoT distributed cloud system for encrypted IoT data that uses a proxy re-encryption scheme that allows the data to only be visible to the owner and any persons present in the smart contract.
  \\\bottomrule\bottomrule
  \caption{Examples of blockchain cloud storage systems~\cite{sharma_blockchain_2021} }
  \label{tab:blockhain cloud storage}
\end{longtable}
\normalsize

\paragraph*{Gaps}
One gap found when researching these solutions was that few offered file versioning that would allow a user to view previous versions of uploaded data. File versioning is a particularly important to this project as users will likely all have varying versions of the same software.