
\chapter{Problem Statement}

\section{The Problem}
\label{sec:problem}

Video games are often large and highly popular pieces of software that are typically distributed for developers by a third party platform like Steam or Epic Games. Whilst these platforms provide benefits such as availability, and some social features they have some major downsides that include:
\vspace{1mm}
\begin{enumerate}[label=(\alph*)]
  \item taking a large cut of all revenue, \newline\textit{Steam take a 30\% cut~\cite{marks_report_2019,brown_valve_2021}}
  \item being vulnerable to censorship from governments, \newline\textit{The Chinese version of Steam is heavily censored~\cite{steamdb_steam_2021}}
  \item the user's access to their games is linked to the platform. \newline\textit{If the platform shuts down, the user loses all their games}
\end{enumerate}
\vspace{1mm}

\section{Goals}

The goal of this project is to implement a large-scale distribution platform that will allow game developers to release and continuously update their games on a public network by directly interacting with their users. This is in the aim to provide greater profits to developer's, freedom from censorship, and better digital ownership for the user.

\section{Scope}

This project will be broken down into two distinct components:

\begin{enumerate}
  \item \textbf{On-Chain} This component will consist of a set of Solidity Smart Contracts written for the Ethereum blockchain that will allow users to view metadata about and purchase games. It will be tested using a local test-net like Ganache using TypeScript. It will later be deployed to the Ethereum test-net to showcase the application in a live network.
  \item \textbf{Off-Chain} This component will be what users will actually run. Each user will join a peer-to-peer network in which they can upload and download games off of other users. This will interface with the blockchain to allow users access to game metadata. See Section~\ref{} for details about how this will be tested.
\end{enumerate}

For both of these, a series of acceptance tests, that directly correlate to individual requirements, will be run and include a series of integration tests to show that my application can meet the requirements and goals I set out. A more detailed description is given in Section~\ref{}.
