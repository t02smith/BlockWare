
\section{Ethereum}

Ethereum is a distributed, transaction-based blockchain that allows the deployment of decentralized applications through the use of smart contracts.

\subsection{Smart Contracts}

Each Ethereum blockchain will contain a smart contract that is an executable piece of code, written in Solidity, that is to be ran by every node in the network, using the Ethereum Virtual Machine. Smart contracts provide a concise set of instructions that produces a predictable outcome, such that given the same input will always produce the same result.
As smart contracts are public, every node on the network can see the result of each execution and track things like currency and asset transfers across the network.
\x
Gas is a unit of measurement that is used to specify the computational effort required to execute operations on the Ethereum network. Each transaction must set a limit on the amount of gas can be used during code executing, and this is paid using ether. However, this can lead to smart contracts failing to execute due to \textit{running out of gas}.
By tying the computational effort of a smart contract to ether, the Ethereum network reduces the risk of DoS attacks as an attacker will likely not have the funds to perform such an attack.
\x
Some example use cases of smart contracts are: creating \& distributing digital assets, decentralized gaming, insurance policies, and financial services.

\subsection{Proof-of-stake}

Ethereum was initially a proof-of-work~\cite{mingxiao_review_2017} based blockchain but has since moved over to a proof-of-stake system~\cite{mingxiao_review_2017}, with the idea that it can save energy. The idea behind proof-of-stake is that a user with a larger stake in the network will have a greater chance of mining a block. It uses the idea of coin age, which takes into account the amount of currency a user has and how old it is.

