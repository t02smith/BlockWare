\section{Risk Assessment}
\label{sec:risk-assessment}

\begin{longtable}[ht]{ p{.2\textwidth} p{0.05\textwidth}  p{0.05\textwidth} p{0.05\textwidth} p{0.53\textwidth}}
  \toprule
  \textbf{Risk}
   & \small\textbf{Loss}
   & \small\textbf{Prob}
   & \small\textbf{Risk}
   & \textbf{Mitigation}
  %
  \\\midrule\midrule
  Difficulty with\newline blockchain\newline development
   & 2
   & 3
   & \cellcolor{orange!50} 6
   & \small I will seek advice from my supervisor about how to tackle certain problems and decide on any changes my project might need. I could also use online documentation or forums for support.
  %
  \x
  Personal illness
  & 3
  & 2
  & \cellcolor{orange!50} 6
  & \small Depending on the amount of lost time, I will have to choose to ignore some lower priority requirements. Use of effective sprint planning will help ensure I can produce at least a minimal viable product.
  %
  \x
  Laptop damaged or lost
  & 3
  & 1
  & \cellcolor{green!30} 3
  & \small Thorough use of version control and periodic backups to a separate drive will ensure I always have a relatively recent copy of my work.
  
  I have other devices available to me at home and through the university to continue development. 
  %
  \x
  The application is not finished
   & 2.5
   & 4
   & \cellcolor{red!30} 10
   & \small Effective use of agile development and requirement prioritisation will ensure that even if I do not complete the project I will have the most significant parts of it developed.

   It is important to consider a cut off point for development, where I will have to purely focus on the write-up and final testing.
  \\\bottomrule\bottomrule
  \caption{The risk assessment of this project}
  \label{tab:risk assessment}
\end{longtable}