% chktex-file 24
% chktex-file 8

\subsection*{Data}
\label{subsec:design-data}

The first consideration is what kind of data we are going to be storing and where is it going to be stored.

\begin{longtable}{ p{.12\textwidth} p{.1\textwidth} p{.1\textwidth} p{.63\textwidth} }
  \toprule
  \textbf{Data} & \textbf{Size} & \textbf{Location} & \textbf{Explanation}\\
  \midrule\midrule
  Game Metadata
  & 100 -- \newline200B
  & Ethereum
  & This data is the minimal set of information required for the unique identification of each game. See Section~\ref{subsubsec:eth-data}.

  \vspace{1mm}
  This data is appropriate to store on Ethereum as it is public, small in size, and essential to the correct functioning of the application as all users will need to be able to discover all games. 
  \x
  Game Hash Tree
  & \~15KB
  & IPFS
  & This will be the compressed Hash Tree that will allow the users to identify and verify the shards of data they need to download for their game.

  \vspace{1mm}
  This data would be costly to store on Ethereum for a large number of games and will only need to be accessed by a subset of users. As it is also public data, IPFS is appropriate to store it on, and we can reference the CID within the data stored on Ethereum.
  \x
  Game Assets
  & Unkown~\footnote{Some games may include many promotional materials, whilst some could include none. Therefore, it is hard to estimate the expected size.} 
  & IPFS
  & This will represent any promotional material provided for the game that can be viewed on the game's store page. This will typically include cover art and a markdown file for the description.

  \vspace{1mm}
  Similar to the Hash Tree, this will typically be too large to store on Ethereum so, given that it is public and non-essential data, IPFS will be used to store and distribute it. 
  \x
  Game Data
  & \textit{avg. 44GB~\footnote{Calculated based off of the top 30 games from \url{https://steamdb.info/charts/} on 22/03/2022}}
  & Peers
  & This will the data required to play the game and will be fetched based upon the contents of the game's Hash Tree.

  \vspace{1mm}
  This data is way too large to store on Ethereum but also isn't public, which means using IPFS would not be appropriate~\footnote{IPFS and similar platforms provide no access control for the data stored there and any encryption based technique would be unviable.}. Therefore, this project will use a custom P2P network for sharing data, which is described in Section~\ref{subsec:design-p2p} 
  \\\bottomrule\bottomrule
  \caption{The different types of data required for each game.}
\end{longtable}

\noindent 
Swarm~\cite{hartman_swarm_1999} was considered as a decentralised storage and distribution platform over IPFS but was decided against as it would couple this project more tightly with Ethereum. On top of that, IPFS has much greater adoption and is much more mature in terms of working on a large scale.