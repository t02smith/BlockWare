\section{Sprint Plans}\label{sec:sprints}

By dividing my implementation into sprints, I was able to incrementally build upon my project by completing requirements according to their priority and expected difficulty. Separating my implementation into sprints benefitted me by:

\begin{itemize}
  \item having a smaller set of requirements to focus on at once helped me to feel less overwhelmed,
  \item working on the most important aspects first to ensure I was able to produce a minimum viable product, and
  \item taking time between sprints to take a break from the project helped reduce burnout and allowed me to prepare for the next sprint.
\end{itemize}

\newparagraph
The following sections will detail each of my sprints, including a breakdown of the requirememnts that were and weren't complete.

\subsection*{Sprint 1}

I anticipated that the P2P game distribution network would be the most complex and time consuming set of requirements in this project so I decided to focus on it for this first sprint. Table~\ref{tab:sprint-1} shows the requirements included for Sprint 1.

\small
\begin{longtable}{p{0.12\textwidth} p{.13\textwidth} p{.65\textwidth}}
  \toprule
  \textbf{Req.} & \textbf{Complete} & \textbf{Evidence/Reasoning}
  \\\midrule\midrule
  \reqref{F-M7}
  & \yes
  & Unit tests for the \textit{model/net/tcp} package and the peer count benchmark tests.
  \\
  \reqref{F-M8}
  & \yes
  & Unit tests for the \textit{model/net/peer/message\_handlers} file test the handling of structured messages and the structured responses sent back. 
  \\
  \reqref{F-M9}
  & \yes
  & The benchmark test show the downloading of data to a large scale.
  \\
  \reqref{F-M10}
  & \yes
  & Unit tests to show incorrect messages being rejected.
  \\
  \reqref{F-M11}
  & \yes
  & User walkthrough 2 shows the download of a game in its entirety. 
  \\
  \reqref{F-M12}
  & \started
  & The algorithm to generate hash trees and the ability to use them to download data was implemented. Unit tests for the \textit{model/manager/hashtree} and \textit{model/manager/games} packages show this. Uploading this to distributed storage was planned for Sprint 2.
  \\\midrule\midrule
  \reqref{NF-M2}
  & \yes
  & User walkthrough 2 shows that any user can establish a connection with any other user.
  \\
  \reqref{NF-S1}
  & \started
  & Users can perform many concurrent connections and channels are used at component boundaries to allow for multiple producers/consumers of data. This requirement was a consideration throughout all sprints.
  \\\bottomrule\bottomrule
  \caption{Requirements included for Sprint 1}
  \label{tab:sprint-1}
\end{longtable}
\normalsize

\subsection*{Sprint 2}

Sprint 2 was about increasing the scope of the application by focusing on two main aspects:

\begin{enumerate}
  \item The integration with Ethereum using a Smart Contract, and
  \item Allowing users to interface with the application via a GUI.
\end{enumerate}

\newparagraph
This sprint had a much slower start compared to the first one as I was largely unfamiliar with smart contract development and the related packages needed to interface with them. On top of this, I considered several UI framework's before settling on my final choice which increased the length of this sprint.
\x
Table~\ref{tab:sprint-2} shows the requirements pitched for Sprint 2.

\small
\begin{longtable}{p{0.12\textwidth} p{.13\textwidth} p{.65\textwidth}}
  \toprule
  \textbf{Req.} & \textbf{Complete} & \textbf{Evidence/Reasoning}
  \\\midrule\midrule
  \reqref{F-M1}
  & \yes
  & Unit tests for the Library smart contract and user walkthrough 1 show the ability to upload game metadata to Ethereum.
  \\
  \reqref{F-M2}
  & \yes
  & Unit tests for the Library smart contract and user walkthrough 4 show the ability to upload an update to an existing game to Ethereum.
  \\
  \reqref{F-M3}
  & \yes
  & Unit tests for the Library smart contract show users of an existing game being given ownership of an updated version.
  \\
  \reqref{F-M4}
  & \yes
  & The smart contract was successfully deployed the Sepolia test-net~\cite{etherscanio_library_nodate}. All user walkthroughs will form connections to this smart contract.\\
  \reqref{F-M5}
  & \yes
  & Unit tests for the Library smart contract and user walkthrough 1 show the successful purchase of a game.
  \\
  \reqref{F-M6}
  & \yes
  & Unit tests for the Library smart contract show a user being added to a mapping containing all users who have purchased the game.
  \\
  \reqref{F-M12}
  & \yes
  & Hash trees are now uploaded to IPFS and the CID is stored on Ethereum. 
  \\
  \reqref{F-S2}
  & \started
  & Basic pages were added according to Section~\ref{subsubsec:frontend}. These pages had little styling or reactivity but could perform the required basic functions. See Appendix~\ref{app:screenshots} for screenshots of the final versions.
  \\\midrule\midrule
  \reqref{NF-M1}
  & \yes
  & The use of the Ethereum blockchain means that no single user can control what is uploaded to the network.
  \\
  \reqref{NF-M3}
  & \yes
  & Developers can be uniquely identified using their Ethereum address. This should be made publically verifiable by the developers.
  \\
  \reqref{NF-M4}
  & \yes
  & Data stored on Ethereum is inherently immutable.
  \\
  \reqref{NF-M5}
  & \yes
  & Unit tests for the Library smart contract show the restriction that only the original uploader can release an update.
  \\\bottomrule\bottomrule
  \caption{Requirements included for Sprint 2}
  \label{tab:sprint-2}
\end{longtable}
\normalsize

\subsection*{Sprint 3}

This sprint was about extending the minimum viable application reached by the end of Sprint 2 with some necessary additions. Table~\ref{tab:sprint-3} shows the list of requirements for this sprint.

\small
\begin{longtable}{p{0.12\textwidth} p{.13\textwidth} p{.65\textwidth}}
  \toprule
  \textbf{Req.} & \textbf{Complete} & \textbf{Evidence/Reasoning}
  \\\midrule\midrule
  \reqref{F-S1}
  & \yes
  & Users will validate each other's Ethereum address after forming a connection and unit tests for the \textit{model/net/peer/message\_handlers} file show this being performed.
  \\
  \reqref{F-S2}
  & \yes
  & The UI was overall improved to improve the user experience.
  \\
  \reqref{F-S3}
  & \no
  & Users will track the blocks sent to them by each of their peers but this application has no mechanism for redeeming these. Due to time constraints, I was unable to implement a sufficient solution. Moreover, I felt that a micro-payment system, like present in Swam~\cite{hartman_swarm_1999}, would be a much better implementation.
  \\
  \reqref{F-S4}
  & \yes
  & Users will exchange the REQ\_PEERS/PEER commands to discover neighbouring peers.\newline
  However a better implementation might have the developer of the game be able to provide a list of peers who have the game. This would allow a user to easily find peers who are interested in the same content.
  \\
  \reqref{F-C1}
  & \no
  & Due to time constraints I was unable to implement this at all.
  \\
  \reqref{F-C2}
  & \yes
  & Game assets are uploaded to IPFS and the CID is stored with the game metadata on Ethereum.
  \\\midrule\midrule
  \reqref{NF-S1}
  & \yes
  & Benchmark tests show the scalability of my application by varying certain parameters and that the target file size can be downloaded within an acceptable best-case.
  \\
  \reqref{NF-S2}
  & \yes
  & Changes to the UI made it more interactive and easier to navigate. Designs were inspired by pages from existing platforms to make the UI feel familiar. See Appendix~\ref{app:screenshots} for screenshots of the final versions.
  \\
  \reqref{NF-C1}
  & \yes
  & A help page was included answering some questions that new users may have about the application.
  \\\bottomrule\bottomrule
  \caption{Requirements included for Sprint 3}
  \label{tab:sprint-3}
\end{longtable}
\normalsize