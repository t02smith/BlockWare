
\section{Benefits}\label{des:benefits}

This application presents the following benefits when compared with centralised game marketplaces:

\begin{itemize}
  \item \textbf{Privacy} A user's personal information and usage isn't collected. Traditional platforms require users to enter personal information and will actively collect data about a user's actions through the platform.
  \item \textbf{Ownership} A user's ownership of a game isn't tied to a single platform and use of Ethereum means that a user's ownership is upheld by all computers in the network.
  \item \textbf{Censorship} Similar to the previous point, no one party has control over the platform so it is much harder for third parties, such as governments, to restrict the content uploaded to it.
  \item \textbf{Profits} Developers and gamers communicate directly and this means developer's won't have to pay a hefty fee for a middle-man. This will result in potentially larger profits for the developers.  
\end{itemize}

\section{Limitations}\label{sec:design-lim}

This application presents the following limitations when compared with a centralised game marketplace:

\begin{itemize}
  \item \textbf{No Social Features} Social features, such as friends or achievements, were not included within the scope of this project.
  \item \textbf{Availability} Section~\ref{subsec:availability} highlights the issue of availability within P2P file-sharing systems and it is likely this platform will face similar issues.
  The use of a contribution system was implemented to help identify those users who have been contributing but there is no automatic rewards system\footnote{An example would be how the micro-payment system works in Swarm~\cite{hartman_swarm_1999}}.
  \item \textbf{Inefficient Updates} As updates are treated as individual games, they will require users to download the entire game again. This is highly inefficient and results in lots of duplicate data being downloaded.
  \item \textbf{Hash Trees} Modelling data as a hash tree means that each file will need at least one block so a game may have a large amount of blocks for not a lot of data and as each block has to be requested separately this will add a lot network overhead. 
\end{itemize}