\section{Integration Testing}

The purpose of integration tests were to evaluate how the application fared when interacting with various types of peer. Each \textit{profile} will mimic a type of behaviour that could be expected in a real-world deployment and the different types are detailed in Table~\ref{tab:profiles}.

\small
\begin{longtable}{p{.15\textwidth} p{.7\textwidth}}
  \toprule
  \textbf{Name} & \textbf{Purpose}
  \\\midrule\midrule
  \textbf{Listen Only}
  & A peer who will listen and respond to all requests perfectly but will never request anything. This is useful for testing when we want a lightweight client to just download data off of. 
  
  This will also be able to upload a game to the locally running smart contract.
  \\
  \textbf{Sender}
  & This peer will respond to requests but will then also reflect the same request back. This shows a more realistic relationship between two peers where they both send and receive data.
  \\
  \textbf{Unreliable}
  & This peer will pseudo-randomly not respond to messages. This is used to show how my application handles requests that sometimes aren't replied to from a peer.
  \\
  \textbf{Selfish}
  & This peer will send requests but will never respond to any. This is used to show how my application handles requests that are never replied to from a peer.
  \\\bottomrule\bottomrule
  \caption{The different profiles used to simulate real-world peers.}
  \label{tab:profiles}
\end{longtable}
\normalsize

\noindent
The main outcome from these different profiles shows the need for a reputation system where a user can distinguish between peers that reliably respond to requests and peers that don't. This would help mitigate some of the overhead of timed-out requests or receiving incorrect data.
