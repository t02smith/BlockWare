\section{Sprint Plans}\label{sec:sprints}

The use of the Agile Methodology~\cite{ilieva_analyses_2004} with sprints was essential in managing my time and ensuring that I was working on the most important aspects of my project first. The use of MoSCoW prioritisation and by then dividing my requirements into logical groups I was able to effectively target key aspects of my application in bulk. The use of test-driven development~\cite{beck_test-driven_2003} meant that at the end of each sprint, each piece of code I wrote was tested and I could move on.
\x
For each sprint we will detail the planned requirements, whether they were completed or not, as well as any general comments about that sprint.

\subsection*{Sprint 1}

I anticipated that the P2P game distribution network would be the most complex and time consuming set of requirements in this project so I decided to focus on it for this first sprint. Table~\ref{tab:sprint-1} shows the requirements included for Sprint 1 and whether they were completed or not.
\x
This sprint was largely problem-free as I didn't have much to learn to be able to complete this and could rely heavily on my design to structure my implementation.

\small
\begin{longtable}{p{0.12\textwidth} p{.13\textwidth} p{.6\textwidth}}
  \toprule
  \textbf{Req.} & \textbf{Complete} & \textbf{Evidence/Reasoning}
  \\\midrule\midrule
  \reqref{F-M7}
  & \yes
  & Unit tests for the model/net/tcp package and the peer count benchmark tests.
  \\
  \reqref{F-M8}
  & \yes
  & Unit tests for the model/net/peer/message\_handlers file test the handling of structured messages and the structured responses sent back. 
  \\
  \reqref{F-M9}
  & \yes
  & All benchmark tests show the downloading of data to a large scale.
  \\
  \reqref{F-M10}
  & \yes
  & Unit tests to show incorrect messages being rejected.
  \\
  \reqref{F-M11}
  & \yes
  & User walkthrough shows the download of a game in its entirety. 
  \\
  \reqref{F-M12}
  & \started
  & The algorithm to generate a hash tree and the using of it to download data was implemented but no way to upload it anywhere.
  \\\midrule\midrule
  \reqref{NF-M2}
  & \yes
  & User walkthrough ... shows that any user can establish a connection with any other user.
  \\
  \reqref{NF-S1}
  & \started
  & Users will form many connections concurrently and optimisations were made using the producer/consumer pattern to complete actions like inserting data, or reqeusting data.
  \\\bottomrule\bottomrule
  \caption{Requirements included for Sprint 1}
  \label{tab:sprint-1}
\end{longtable}
\normalsize

\subsection*{Sprint 2}

Sprint 2 was about increasing the scope of the application by focusing on two main aspects:

\begin{enumerate}
  \item The integration with Ethereum using a Smart Contract, and
  \item Allowing users to interface with the application via a GUI.
\end{enumerate}

\vspace{2mm}\noindent
This sprint had a much slower start compared to the first one as I was largely unfamiliar with smart contract development and the related packages needed to interface with them. On top of this, I considered several UI framework's before settling on my final choice which increased the length of this sprint.
\x
Table~\ref{tab:sprint-2} shows the requirements pitched for Sprint 2 and whether or not they were completed.

\small
\begin{longtable}{p{0.12\textwidth} p{.13\textwidth} p{.6\textwidth}}
  \toprule
  \textbf{Req.} & \textbf{Complete} & \textbf{Evidence/Reasoning}
  \\\midrule\midrule
  \reqref{F-M1}
  & \yes
  & Unit tests for the Library smart contract and user walkthrough ... show the ability to upload game metadata to Ethereum.
  \\
  \reqref{F-M2}
  & \yes
  & Unit tests for the Library smart contract and user walkthrough ... show the ability to upload an update to an existing game to Ethereum.
  \\
  \reqref{F-M3}
  & \yes
  & Unit tests for the Library smart contract show users of an existing game being given ownership of an updated version.
  \\
  \reqref{F-M4}
  & \yes
  & The smart contract was successfully deployed the Sepolia test-net~\cite{etherscanio_deployed_nodate}. All user walkthroughs will form connections to this smart contract.\\
  \reqref{F-M5}
  & \yes
  & Unit tests for the Library smart contract and user walkthrough ... show the successful purchase of a game.
  \\
  \reqref{F-M6}
  & \yes
  & Unit tests for the Library smart contract show a user being added to a mapping containing all users who have purchased the game.
  \\
  \reqref{F-M12}
  & \yes
  & Hash trees are now uploaded to IPFS and the CID is stored on Ethereum. 
  \\
  \reqref{F-S2}
  & \started
  & Basic pages were added according to Section~\ref{subsubsec:frontend}. These pages had little styling or reactivity but could perform the required basic functions. See Appendix~\ref{app:screenshots} for screenshots of the final versions.
  \\\midrule\midrule
  \reqref{NF-M1}
  & \yes
  & The use of the Ethereum blockchain means that no single user can control what is uploaded to the network.
  \\
  \reqref{NF-M3}
  & \yes
  & Developers can be uniquely identified using their Ethereum address. This should be made publically verifiable by the developers.
  \\
  \reqref{NF-M4}
  & \yes
  & Data stored on Ethereum is inherently immutable.
  \\
  \reqref{NF-M5}
  & \yes
  & Unit tests for the smart contract show the restriction that only the original uploader can release an update.
  \\\bottomrule\bottomrule
  \caption{Requirements included for Sprint 2}
  \label{tab:sprint-2}
\end{longtable}
\normalsize

\subsection*{Sprint 3}

This sprint was about extending the minimum viable application reached by the end of Sprint 2 with some necessary additions. Table~\ref{tab:sprint-3} shows the list of requirements for this sprint and whether or not they were completed.
\x
I definitely found this sprint to be the most difficult for a number of reasons:

\begin{enumerate}
  \item As I was beginning to test my application with other simulated applications, I kept coming across bugs that were hard to locate and replicate, which made fixing them tedious.
  \item I didn't have as much free time to spend working on this project due to the other modules I was taking at the time.
  \item Some of the requirements pitched introduced relatively complex mechanics that I was not entirely satisfied with but didn't have the time to redesign them.  
\end{enumerate}

\small
\begin{longtable}{p{0.12\textwidth} p{.13\textwidth} p{.6\textwidth}}
  \toprule
  \textbf{Req.} & \textbf{Complete} & \textbf{Evidence/Reasoning}
  \\\midrule\midrule
  \reqref{F-S1}
  & \yes
  & Users will validate each other's Ethereum address after forming a connection and unit tests for the model/net/peer/message\_handlers file show this being performed.
  \\
  \reqref{F-S2}
  & \yes
  & The UI was overall improved to improve the user experience.
  \\
  \reqref{F-S3}
  & \no
  & Users will track the blocks sent to them by each of their peers but this application has no mechanism for redeeming these. Due to time constraints, I was unable to implement a sufficient solution. Moreover, I felt that a micro-payment system, like present in Swam~\cite{hartman_swarm_1999}, would be a much better implementation.
  \\
  \reqref{F-S4}
  & \yes
  & Users will exchange the REQ\_PEERS/PEER commands to discover neighbouring peers.\newline
  However a better implementation might have the developer of the game be able to provide a list of peers who have the game. This would allow a user to easily find peers who are interested in the same content.
  \\
  \reqref{F-C1}
  & \no
  & Due to time constraints I was unable to implement this at all.
  \\
  \reqref{F-C2}
  & \yes
  & Game assets are uploaded to IPFS and the CID is stored with the game metadata on Ethereum.
  \\\midrule\midrule
  \reqref{NF-S1}
  & \yes
  & Benchmark tests show the scalability of my application by varying certain parameters and that the target file size can be downloaded within an acceptable best-case.
  \\
  \reqref{NF-S2}
  & \yes
  & Changes to the UI made it more interactive and easier to navigate. Designs were inspired by pages from existing platforms to make the UI feel familiar. See Appendix~\ref{app:screenshots} for screenshots of the final versions.
  \\
  \reqref{NF-C1}
  & \no
  & Completing this requirement would be incredibly complex and was decided against being completed. Preventing the distribution of illegal content is an important consideration moving forward to help keep the platform safe.
  \\
  \reqref{NF-C2}
  & \yes
  & A help page was included answering some questions that new users may have about the application.
  \\\bottomrule\bottomrule
  \caption{Requirements included for Sprint 3}
  \label{tab:sprint-3}
\end{longtable}
\normalsize