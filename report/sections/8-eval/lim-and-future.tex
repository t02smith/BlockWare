\section{Limitations and Future Work}

\subsection{Limitations}

\subsection{Future Work}

\subsubsection*{Content Discovery}
Creating a fully functioning store page, where users can search for and discover new games, would be a vital next step for this application. Some of the techniques used to achieve this could be:

\begin{itemize}
  \item \textbf{Indexing} Periodically generate an index of all games uploaded that allows users to easily search the store without having to make large amounts of requests to the blockchain. Use of ranking algorithms could allow for users to be shown the most relevant and useful results.
  \item \textbf{More Metadata} Adding more metadata to games would allow for them to be more easily searchable and indexable. For example, each game might be given a set of tags that can be searched for.
\end{itemize}

\subsubsection*{Optimisations}

\paragraph*{New Commands}
Extending the command set from Section~\ref{subsubsec:commands} with more complex ones (such as batch requests) could potentially reduce the number of interactions two peers would need to make and remove a lot of overhead gained from exchanging lots of commands.

\paragraph*{UDP Over TCP}
Currently each user manages a set of TCP connections with their peers and this creates a lot of overhead from having to maintain channels of communication that may not necessarily be used. A UDP approach would allow for faster communication with less network overhead at the expense of reliability and greater complexity.

\paragraph*{Block Selection}
Currently blocks are not ranked in any way but by considering ideas from Section~\ref{subsec:bittorrent-download} we could improve the efficiency, availability and throughput of our network.

\subsubsection*{New Features}

\paragraph*{Automated Rewards}
Currently there is no system in place to automatically reward users for contributing. A micro-payment system like present in Swarm~\cite{hartman_swarm_1999} could be used to give a guaranteed incentive to users.