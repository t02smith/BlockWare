
\subsection{Persistence}

The Persistence layer shows how the data for the application is divided across several mediums; namely the \textbf{Ethereum Smart Contract}, \textbf{IPFS}, and a \textbf{P2P Network}. Each component stores a different kind of data and the justification for these platforms is given below.

\subsubsection*{Ethereum}\label{subsubsec:impl-eth}

An Ethereum Smart Contract, written in Solidity \url{https://docs.soliditylang.org/en/v0.8.19/}, will be used to store the set of data about games that is required for the identification of each game, see Section~\ref{subsubsec:eth-data}. The Smart Contract will also be used to perform the following:

\begin{enumerate}
  \item \textbf{Purchasing Games} Users will purchase games over the Ethereum network and will have their address added to the set of users who own a given game. This means all users can see whether a given peer owns a game before sending them the data.
  \item \textbf{Uploading a Game} Users will need to be able to upload a game to the Ethereum network and have it be visible and purchasable by all users of the application.
\end{enumerate}

\vspace{2mm}\noindent
The go-ethereum package \url{https://geth.ethereum.org/} will allow us to interact with the ethereum blockchain and Abigen \url{https://docs.avax.network/specs/abigen} will allow us to compile any smart contracts to Go code. This will allow us to interact with our smart contract on ethereum using a set of Go functions.


\subsubsection*{IPFS}

Storing data on Ethereum is costly to the uploader, so it is important that the data we do store on it is important and minimal. However we still need to be able to store the following pieces of data such that they are publicly available:

\begin{itemize}
  \item \textbf{Hash Tree} The tree representation of the game's data contents. Only users who purchase the game will need this.
  \item \textbf{Assets} Digital assets such as artwork, description, promotional work, etc.  
\end{itemize}

\vspace{2mm}\noindent
This project will use the IPFS implementation Kubo \url{https://github.com/ipfs/kubo}, due to it being the most widely used implementation of IPFS. We will use the go-ipfs-api library \url{https://github.com/ipfs/go-ipfs-api} to interact with Kubo and upload/download the data specified above.

\subsubsection*{Game Data}

One limitation to IPFS is that the only way to store private data on it is to encrypt it. This means it would be difficult to manage and control access rights of data on IPFS over a large, distributed network where only a subset of users will actually be allowed access to it. 
On top of this, we would not be able to track the contributions of individual nodes sharing data such that a game developer could reward a user for their contribution. 
\x
The lack of middleware support by IPFS shows that, to meet the requirements set out in Section~\ref{subsec:requirements}, we would need to develop a separate platform to share game data over.
The details of this network and how data is managed within it is detailed in Section~\ref{subsec:backend}
