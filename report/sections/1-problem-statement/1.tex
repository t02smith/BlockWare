
\chapter{Problem Statement}

\section{The Problem}\label{sec:problem}

Video games are large, highly popular pieces of software that are typically obtained through centralised platforms, like Steam or Epic Games. These platforms offer high availability, social features and customer support at the expense of:

\begin{itemize}
  \item taking a large cut of revenue from developers\footnote{Steam take a 30\% cut~\cite{marks_report_2019,brown_valve_2021}},
  \item being prone to censorship from entities like governments\footnote{See the Chinese version of Steam~\cite{noauthor_steam_nodate-1}}, and
  \item relying on a single platform to stay active, distribute games and maintain a user's ownership\footnote{For example, the Nintendo eShop closing down~\cite{noauthor_nintendo_2022}}.
\end{itemize}

\section{Goals}

This project aims to produce a proof-of-concept, decentralised video game marketplace, using blockchain technology, that will allow developers to continuously release and update their games on a public network where they directly interact with their users. This network should be built on top of a peer-to-peer file-sharing network to allow users to to distribute games.

\section{Scope}

This project will consist of the following components:

\begin{enumerate}
  \item An Ethereum smart contract that will allow us to maintain a library of games that can be queried and added to by any user.
  \item A local application to be run by users to interact with the smart contract and allow users to join a peer-to-peer network where they can download and upload games to.
\end{enumerate}