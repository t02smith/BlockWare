
\section{Blockchain-Based Cloud Storage}

Blockchain technology can be leveraged for distributed cloud storage to provide both public and private storage. Table~\ref{tab:blockhain cloud storage} details some examples of how blockchain has been used to create cloud storage platforms:

\small
\begin{longtable}{ p{0.35\textwidth} p{0.6\textwidth} }
  \toprule
  \textbf{Paper} & \textbf{Description of Solution}
  \\\midrule\midrule
  Blockchain Based Data Integrity Verification in P2P Cloud Storage~\cite{yue_blockchain_2018}
  & This paper looks at how varying structures of Merkle Trees can be used to verify the integrity of data within a P2P blockchain cloud storage system.
  %
  \x
  Deduplication with Blockchain for Secure Cloud Storage~\cite{li_deduplication_2018}
  & A deduplication scheme that uses the Ethereum blockchain to record storage information and distribute files across multiple servers
  %
  \x
  Block-secure: Blockchain based scheme for secure P2P cloud storage~\cite{li_block-secure_2018}
  & Users divide their own data into encrypted blocks and upload them randomly into a blockchain, P2P network. It uses a custom genetic algorithm to solve the file block replica placement problem and ensure data availability.
  %
  \x
  Blockchain-Based Medical Records Secure Storage and Medical Service Framework~\cite{chen_blockchain-based_2018}
  & Describes a secure and immutable storage scheme to manage personal medical records as well as a service framework to allow for the sharing of these records.
  %
  \x
  A Blockchain-Based Framework for Data Sharing With Fine-Grained Access Control in Decentralized Storage Systems~\cite{wang_blockchain-based_2018}
  & This solution proposes an attribute-based encryption scheme that allows the distribution of access keys to users based upon their allocated groups. Data is uploaded to IPFS and uses an Ethereum smart contract to implement a keyword search of the data stored.
  %
  \x
  Blockchain based Proxy Re-Encryption Scheme for Secure IoT Data Sharing~\cite{manzoor_blockchain_2019}
  & An IoT distributed cloud system for encrypted IoT data that uses a proxy re-encryption scheme that allows the data to only be visible to the owner and any persons present in the smart contract.
  \\\bottomrule\bottomrule
  \caption{Examples of blockchain cloud storage systems~\cite{sharma_blockchain_2021} }
  \label{tab:blockhain cloud storage}
\end{longtable}
\normalsize

\paragraph*{Gaps}
Some of the gaps found include:
\begin{itemize}
  \item \textbf{Access Control} Many of these include no dynamic access-control system, and~\cite{wang_blockchain-based_2018} requires the owner of the data to manually manage access rights. 
  \item \textbf{Versioning} Many of these systems rely on platforms like IPFS, which do not natively allow for versioning without uploading a new set of data entirely.
\end{itemize}

\newparagraph
This project will allow for dynamic access control through a payment system based on Ethereum and will allow for pieces of data to be versioned to match a more realistic lifecycle for a piece of software.