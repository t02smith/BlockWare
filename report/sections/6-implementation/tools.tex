
\section{Backend}

The backend code for this application was written based upon the design from Section~\ref{subsec:backend} and was done using the following tools:

\begin{longtable}{p{0.15\textwidth} p{0.7\textwidth}}
  \toprule
  \textbf{Tool} & \textbf{Description \& Reasoning}
  \\\midrule\midrule
  Go
  & Go was chosen because of its simple syntax, high performance, strong standard library and third party packages for interacting with Ethereum.
  \\
  go-ipfs-api
  & A Go package used for interacting with the Kubo implementation of IPFS that gave an easy interface for downloading and uploading data to Kubo.\\
  go-ethereum
  & A collection of tools used for interacting with Ethereum including an Ethereum CLI client Geth, and a tool for converting Ethereum contracts into Go packages.
  \\\bottomrule\bottomrule
  \caption{The tools used to develop the backend}
\end{longtable}

\section{Smart Contract}

To write and deploy a smart contract that met the criteria specified in Section~\ref{subsec:design-con-eth}, I used the following tools:

\begin{longtable}{p{0.15\textwidth} p{0.7\textwidth}}
  \toprule
  \textbf{Tool} & \textbf{Description \& Reasoning}
  \\\midrule\midrule
  Solidity
  & The language used to write smart contracts for the Ethereum blockchain.\\
  Sepolia
  & An Ethereum test-net that used to deploy my smart contract to. One of the main benefits was that it provides a fast transaction time for quick feedback.\\
  Alchemy
  & Alchemy provides useful tools for interacting with Ethereum and specifically Sepolia, such as an ETH faucet and an RPC URL.\\
  MetaMask
  & A browser-based wallet that can easily be conencted to other tools such as Alchemy or Remix.\\
  Remix
  & A browser-based IDE for writing smart contracts that allows for easy deployment.
  \\\bottomrule\bottomrule
  \caption{The tools used for deployment of my smart contract}
\end{longtable}

\vspace{2mm}\noindent
The contract was successfully deployed to Sepolia and the details of it can be viewed at \url{https://sepolia.etherscan.io/address/0x2899dab55a4a20d698062bbf4d4ce9f1073ce052}.

\section{Other Tools}

The following tools were also used throughout development:

\begin{longtable}{p{0.15\textwidth} p{0.7\textwidth}}
  \toprule
  \textbf{Tool} & \textbf{Description \& Reasoning}
  \\\midrule\midrule
  Git
  & A version control system used in conjunction with GitHub. Creating periodic commits meant I always had a recent backup available and could easily backtrack to help find issues. Use of a GitHub Actions helped remind me that not all of my tests passed at all times :(.\\
  LaTeX
  & Used for the write-up of this document. LaTeX was useful in creating a large document and has many packages that help with referencing and design.\\
  VSCode
  & My code editor of choice for this project as it allowed me to seamlessly work on both my frontend and backend code at once.\\
  Lucidchart
  & Lucidchart was used to create all of the diagrams for this project.
  \\\bottomrule\bottomrule
  \caption{General purpose tools used for this project}
\end{longtable}

% \begin{longtable}[ht]{|m{.15\textwidth}|m{.15\textwidth}|p{.65\textwidth}|}
%   \hline
%   \hdr{Component} 
%   & \hdr{Name}
%   & \hdr{Description \& Reasoning}  \\ \hline
%   \multirow{2}{.15\textwidth}{Backend} 
%   & Golang
%   & \url{https://go.dev/}\newline Google's Go was used due to its simple syntax, high performance, strong standard library and third party packages for interacting with the Ethereum blockchain.
%   \\ \cline{2-3} 
%   & go-ipfs-api
%   & \url{https://github.com/ipfs/go-ipfs-api}\newline Used for interacting with the Kubo IPFS instance for downloading/updating data.
%   \\ \hline
%   Controller
%   & \multirow{2}{.15\textwidth}{Wails}
%   & \multirow{2}{.65\textwidth}{Wails allowed me to write a webkit frontend for my Golang backend code using a modern reactive UI framework. It allows you to write controller functions in Go that can be called by the frontend.}\newline
%   \\\cline{1-1}
%   \multirow{4}{.15\textwidth}{Frontend}
%   & 
%   & \newline
%   \\\cline{2-3}
%   & Vue.js v3
%   & \url{https://vuejs.org/}\newline A components based web-framework that offers great state and components lifecycle hooks to create a reactive and complex user interface.
%   \\\cline{2-3}
%   & Pinia
%   & \url{https://pinia.vuejs.org/}\newline The recommended state management tool for Vue.js. It allows us to simplify Vue components and improve the usability of them.
%   \\\cline{2-3}
%   & SCSS
%   & \url{https://sass-lang.com/}\newline An extension of CSS for styling my Vue.js components that includes lots of quality of life improvements. 
%   \\\hline
%   \caption{The list of tools used during development}
% \end{longtable}

