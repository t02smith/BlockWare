
\section{P2P File Sharing}
\label{sec:lit-p2p}

Table~\ref{tab:lit-review-p2p} looks at various implementations of P2P file-sharing networks. Using the pros and cons of each will help me identify key considerations of my own implementation.

\small
\begin{longtable}{ p{0.15\textwidth} p{0.8\textwidth} }
  \toprule
  \textbf{System} & \textbf{Description of Solution}
  \\\midrule\midrule
  IPFS~\cite{benet_ipfs_2014}
  & IPFS is a set of protocols for transferring and organising data over a content-addressable, peer-to-peer network. IPFS is open source and has many different implementations, such as Estuary~\cite{noauthor_estuary_nodate-1} or Kubo~\cite{noauthor_ipfskubo_2023}.
  %
  \x
  Swarm~\cite{hartman_swarm_1999}
  & Swarm is a distributed storage solution linked with Ethereum that has many similarities with IPFS~\cite{pouwelse_bittorrent_2005}. It uses an incentive mechanism, Swap (Swarm Accounting Protocol), that keeps track of data sent and received by each node in the network and then the payment owed for their contribution.
  %
  \x
  BitTorrent~\cite{pouwelse_bittorrent_2005}
  & See Section~\ref{sec:bittorrent}.
  %
  \x
  AFS~\cite{morris_andrew_1986,howard_scale_1988}
  & The Andrew File System was a prototype distributed system by IBM and Carnegie-Mellon University in the 1980s that allowed users to access their files from any computer in the network.
  %
  \x
  Napster~\cite{saroiu_measurement_2001}
  & Napster uses a central cluster of servers that maintain an index of every file on the network and which users have a copy of it. Clients will query the cluster for this information and will choose peers based upon their bandwidth.  
  %
  \x
  Gnutella~\cite{saroiu_measurement_2001}
  & In Gnutella, nodes form an overlay network and will discover other peers through \textit{ping-pong} messages, where any node that receives a \textit{ping} message will forward it to their neighbours and send a \textit{pong} message to the originator. 
  A user will flood a download request to their peers until they find a suitable peer to download off of.
  \\\bottomrule\bottomrule
  \caption{Various global distributed file systems.}
  \label{tab:lit-review-p2p}
\end{longtable}
\normalsize

\paragraph*{Gaps}
The main issues involving these networks are:

\begin{enumerate}
  \item \textbf{Trust} Nodes are typically anonymous so users have to trust that the data they download isn't malicious. Moreover, there are typically no measures included to take down harmful or illegal content.
  \item \textbf{Payment} None of the networks natively support payment for content or include mechanisms for large-scale access control. 
\end{enumerate}

\newparagraph
This project seeks to tie a traditional P2P file-sharing network with the Ethereum blockchain to allow for greater trust through publicly identifiable uploaders and allow users to pay for content and be stored in a public record.