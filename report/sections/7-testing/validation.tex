% chktex-file 44

\section{Acceptance Testing}\label{sec:acc-tests}

A user walkthrough is a series of steps to take that, if completed, prove the completeness of a set of requirements. Table~\ref{tab:walkthroughs} describes all user walkthroughs and Appendix~\ref{app:user-walkthrough} shows the evidence for their completeness.
All user walkthroughs use a contract deployed to the Sepolia test-net~\cite{etherscanio_library_nodate}, satisfying \reqref{F-M4}, \reqref{NF-M2} and \reqref{NF-M1}. I have also uploaded the smart contract source code to demonstrate that specific functions are called following specific events. 

\newcommand{\p}[1]{$P_{#1}$}
\newcommand{\g}[1]{$G_{#1}$}

\small
\begin{longtable}{ p{.02\textwidth} p{.2\textwidth} p{.56\textwidth} p{0.1\textwidth} }
  \toprule
  \textbf{Id} & \textbf{Requirements} & \textbf{Description} & \textbf{Success}\\\midrule\midrule
  1
  & \reqref{F-M1} \reqref{F-M5} \reqref{F-M12} \reqref{F-S2} \reqref{F-C2} \reqref{NF-M3}
  & \vspace{-5mm}\begin{enumerate}[wide, labelwidth=!, labelindent=0pt]
    \item \p{1} uploads a game \g{1}.
    \item \p{2} finds \g{1} on the store.
    \item \p{2} purchases \g{1}.
    \item \p{2} shows \g{1} added to their library.
  \end{enumerate}
  & \yes
  \\\midrule
  2 
  & \reqref{F-M6} \reqref{F-M8} \reqref{F-M9} \reqref{F-M10} \reqref{F-M11} \reqref{F-S1} \reqref{F-S2} \reqref{F-S3} \reqref{NF-M2} 
  & \vspace{-5mm}\begin{enumerate}[wide, labelwidth=!, labelindent=0pt]
    \item \p{2} connects to \p{1}.
    \item \p{1} and \p{2} exchange Ethereum addresses.
    \item \p{2} starts a download for \g{1}.
    \item \p{2} sends requests for blocks to \p{1}.
    \item \p{1} queries the smart contract to verify that \p{2} owns \g{1}.
    \item \p{1} will respond to \p{2} with the requested data.
    \item \p{2} will verify each block of data received using its hash.
    \item \p{2} will have full downloaded \g{1}.
    \item \p{1} will request and receive a contributions receipt for \g{1} from \p{2}.
  \end{enumerate}
  & \yes
  \\\midrule
  3
  & \reqref{F-M2} \reqref{F-M3} \reqref{F-M6} \reqref{NF-M5}
  & \vspace{-5mm}\begin{enumerate}[wide, labelwidth=!, labelindent=0pt]
    \item \p{1} is the original uploader of \g{1} and \p{2} has already purchased \g{1}.
    \item \p{1} uploads an update to \g{1}, \g{2}.
    \item \p{2} will hit the ‘check for updates’ button and see \g{2} in their library.
  \end{enumerate}
  & \yes
  \\\midrule
  4
  & \reqref{F-S4} \reqref{F-M7} \reqref{NF-M2} 
  & \vspace{-5mm}\begin{enumerate}[wide, labelwidth=!, labelindent=0pt]
    \item \p{1} is connected to \p{2}.
    \item \p{3} forms a connection with \p{1}.
    \item \p{3} requests a list of \p{1}'s peers and \p{1} responds with the details for \p{2}.
    \item \p{4} forms connections with \p{2}.
  \end{enumerate}
  & \yes
  \\\bottomrule\bottomrule
  \caption{The set of user walkthroughs used to prove the completeness of this project's requirements.}
  \label{tab:walkthroughs}
\end{longtable}
\normalsize