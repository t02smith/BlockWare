
\section{Design Considerations}

\subsection*{Type of Blockchain}

To satisfy \textbf{NF\_M2} and \textbf{NF\_M1}, we will need to use a public blockchain, which will benefit our project by:

\begin{itemize}
  \item being accessible by a greater amount of people, which should boost availability and scalability (satisfying \textbf{NF\_S1}),
  \item reducing the risk of censorship,
  \item providing greater data integrity (\textbf{NF\_M4})
\end{itemize}

\noindent Ethereum is a public blockchain that allows developers to publish their own distributed applications to it; it comes with an extensive development toolchain so is an obvious choice for this project (\textbf{F\_M8}).


\subsection*{Uploading Content}
\label{subsec:upload-content}

For developers to upload their game (\textbf{F\_M5}), they must provide a digital certificate to prove their identity (\textbf{NF\_M3}) as well as the required metadata (\textbf{F\_M1}) for identifying, downloading and verifying the game. The developer is then expected to allow users to purchase the game off them and seed the game to at least an initial group of users.

\begin{figure}[ht]
  \centering
  \includegraphics[width=.85\textwidth]{images/diagrams/block-body.png}
  \caption{\textit{How shard data is stored.}}
  \label{fig:hash-storage}
\end{figure}

\subsection*{Purchasing Content}

Users will purchase content from developers using Ether (\textbf{F\_M9}) and will be provided with a proof of purchase (\textbf{F\_M10}) that is encrypted with their private key and can be used by any node in the network to verify the purchase.

\subsection*{Downloading Content}

Games will be content addressable, using their root hash stored on the blockchain, and will allow users to discover nodes to download off of (\textbf{F\_M3}). Once a user finds a node it will:

\begin{enumerate}
  \item Send their proof of purchase of the desired game (\textbf{F\_S2}).
  \item request individual shards from the node using the shard's hash (\textbf{F\_M2}),
  \item use the metadata from the blockchain to verify the block's contents (\textbf{F\_M7}),
  \item send a confirmation message that proves the successful transfer of a block (\textbf{F\_S1}), 
  \item and repeat this until the entirety of the game is installed (\textbf{F\_M6}).
\end{enumerate}

\noindent Shards will be downloaded in a similar order to that of BitTorrent, which is described in Section~\ref{subsec:bittorrent-download}.

\subsection*{Updating Content}

To satisfy \textbf{F\_M4}, developers will perform the same steps as in Section~\ref{subsec:upload-content} but will also include the hash of a previous block that contains the older version of the game. This will include the restriction that only the original uploader can upload an update to a piece of software (\textbf{NF\_S2}).

\begin{figure}[ht]
  \centering
  \includegraphics[width=.85\textwidth]{images/diagrams/update-software.png}
  \caption{\textit{How blocks can relate to older blocks.}}
\end{figure}

\noindent It is likely that many shards will persist between versions so a node will only ever download the changed or new data. To satisfy \textbf{F\_C1}, a node may optionally keep older shards that have been removed or changed.


\subsection*{Proving Contribution}

When a user purchases a piece of software they will be granted a unique seeder token. When a user successfully downloads a shard of data off of a peer they will reply with a confirmation message, containing this seeder token, that is encrypted using the developers public key. When a user wants to prove that they have contributed to the distribution of the game, they will send a collection of these messages to the developer, who will judge their validity.

\subsection*{Sequence Diagram}

\begin{figure}[ht]
  \centering
  \includegraphics[width=.95\textwidth]{images/diagrams/seqeunce-diagram.png}
  \caption{\textit{The main interactions within the application.}}
\end{figure}