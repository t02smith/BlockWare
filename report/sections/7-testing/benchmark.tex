% chktex-file 44

\section{Benchmarking}\label{sec:benchmark}

Benchmarking is being used in this project to determine the overall performance and scalability of the application, whilst also being useful in identifying any bottlenecks or how the end user can optimise their inputs. The key benchmark being assessed is related to how the application scales downloading games by varying the following factors:

\begin{enumerate}
  \item how many of the peers we are connected to have the data we need,
  \item how large is the game (in terms of average file size and number of files), and
  \item the shard size used to create the hash tree.
\end{enumerate}

\vspace{2mm}\noindent
All benchmarks will need to be ran on the same machine to ensure consistency.

\subsection*{Number of Peers}

For each run, we will create a project with 500 files, each of size 80MB~\footnote{These numbers were chosen to match our average game size from Section~\ref{subsec:design-data}, where $500\times 80MB = 40GB$} and a shard size of $2^{22}$ = 4MiB. We will then run $N$ peers locally to simulate a perfect network connection.

\begin{longtable}{ | r | r | }
  \hline
  \textbf{Peer Count $N$} & \textbf{Time (ms)} \\\hline
  1
  &
  \\\hline
  2
  &
  \\\hline
  4
  &
  \\\hline
  8
  &
  \\\hline
  \caption{How varying peer count affects download speed}
\end{longtable}

It is clear that the overhead of managing more peers will have an effect on performance. Using UDP over TCP could help us when scaling to a larger number of peers.

\subsection*{Game Size}

For each run, we will create a project with $F$ files of size $S$MB, such that $F\times S = 40GB$, and a shard size of 4MiB. We will then run 1 peer locally to simulate a perfect network connection.

\begin{longtable}{ | r | r | r | r | }
  \hline
  \textbf{File Count $F$} & \textbf{File Size $S$ (MB)} & \textbf{Time (ms)} \\\hline
  100
  & 400
  &
  \\\hline
  50
  & 800
  &
  \\\hline
  25
  & 1,600
  &
  \\\hline
  5
  & 8,000
  &
  \\\hline
  1
  & 40,000
  &
  \\\hline
  \caption{How varying file count and size affects download speed}
\end{longtable}

\subsection*{Shard Size}

For each run we will create a new project with 500 files, each of size 80MB, and a shard size of $B$MiB. We will then run 1 peer locally to simulate a perfect network connection.

\begin{longtable}{ | r | r | }
  \hline
  \textbf{Shard Size $B$ (MiB)} & \textbf{Time (ms)} \\\hline
  1
  &
  \\\hline
  2
  &
  \\\hline
  4
  &
  \\\hline
  8
  &
  \\\hline
  16
  &
  \\\hline
  \caption{How varying the shard size of the hash tree affects download speed}
\end{longtable}