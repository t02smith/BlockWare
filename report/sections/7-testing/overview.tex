\section{Overview}

My approach to testing will consist of the following principles:

\begin{enumerate}
  \item \textbf{Test Driven Development~\cite{beck_test-driven_2003}} Tests should be written alongside the code to reduce the risk of bugs and improve robustness.
  \item \textbf{Fail Fast (Smoke testing)} Automated tests should be ran in a pipeline where the fastest tests are always ran first to reduce the time spent running tests.
  \item \textbf{Documentation} Test cases should be well documented and grouped contextually such that they are easy to maintain and add to.
\end{enumerate}

\subsection*{Tools}

In Table~\ref{tab:tools-testing} I detail the different tools I used to write automated tests for my application.

\begin{longtable}{p{0.15\textwidth} p{0.7\textwidth}}
  \toprule
  \textbf{Tool} & \textbf{Description \& Reasoning}
  \\\midrule\midrule
  Go\newline Testing~\cite{noauthor_testing_nodate}
  & The testing package included with Go's standard library was sufficient to produce most of the test cases required for this project.
  \\
  testify~\cite{noauthor_testify_2023}
  & This package is included as it provides several useful testing features that aren't present in the standard library testing package. This includes assert functions to boost code readability, mocking tools for better unit testing, setup/teardown functionality, and more.
  \\\bottomrule\bottomrule
  \caption{The tools used for testing my project}
  \label{tab:tools-testing}
\end{longtable}
