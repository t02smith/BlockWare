
\section{Blockchain-Based Cloud Storage}\label{sec:lit-blockchain}

This project will need to maintain an immutable and public record across a P2P network such that it searchable and can be trusted. Table~\ref{tab:blockhain cloud storage} shows how blockchain technology has been used to provide a solution distributed cloud storage systems.


\small
\begin{longtable}{ p{0.35\textwidth} p{0.6\textwidth} }
  \toprule
  \textbf{Paper} & \textbf{Description of Solution}
  \\\midrule\midrule
  Blockchain Based Data Integrity Verification in P2P Cloud Storage~\cite{yue_blockchain_2018}
  & The Ethereum blockchain is used to add trust to a data verification system for a P2P network. It analyses how varying structures of Merkle Trees affect the performance of verification of data stored in the network. 
  %
  \x
  Deduplication with Blockchain for Secure Cloud Storage~\cite{li_deduplication_2018}
  & This paper implements a deduplication scheme by uploading storage information to Ethereum and uses smart contract based protocols to provide secure deduplication over encrypted data.
  %
  \x
  Block-secure: Blockchain based scheme for secure P2P cloud storage~\cite{li_block-secure_2018}
  & Users divide their own data into encrypted blocks and upload them randomly into a blockchain, P2P network. It uses a custom genetic algorithm to solve the file block replica placement problem and ensure data availability.
  %
  \x
  Blockchain-Based Medical Records Secure Storage and Medical Service Framework~\cite{chen_blockchain-based_2018}
  & Describes a blockchain-based platform that would allow for the secure and immutable storage of user medical records, such that they are independant from any individual medical institution and users have greater control over who can access their personal data.
  %
  \x
  A Blockchain-Based Framework for Data Sharing With Fine-Grained Access Control in Decentralized Storage Systems~\cite{wang_blockchain-based_2018}
  & An attribute-based encryption scheme that allows the distribution of access keys to users based upon their allocated groups. Data is uploaded to IPFS and an Ethereum smart contract is used to implement a keyword search of the data stored.
  %
  \x
  Blockchain based Proxy Re-Encryption Scheme for Secure IoT Data Sharing~\cite{manzoor_blockchain_2019}
  & A distributed cloud system for encrypted IoT data that uses a proxy re-encryption scheme that allows the data to only be visible to the owner and any persons present in the smart contract.
  \\\bottomrule\bottomrule
  \caption{Examples of blockchain cloud storage systems~\cite{sharma_blockchain_2021} }
  \label{tab:blockhain cloud storage}
\end{longtable}
\normalsize

\noindent
Some common themes found include:

\begin{itemize}
  \item \textbf{Access Control} If these platforms offer access control it is typically up for the owner of the data to distribute access rights. 
  \item \textbf{Data Integrity} The blockchain is primarily used to add trust across a distributed network in the form of immutability or providing data integrity services.
  \item \textbf{Data Location} Data stored on the blockchain is commonly metadata that refers to some data stored elsewhere, typically on a form of distributed storage network. 
\end{itemize}
