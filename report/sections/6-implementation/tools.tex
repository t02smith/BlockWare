
\section{Backend}

The backend code for this application was written based upon the design from Section~\ref{subsec:backend} and was done using the following tools:

\begin{longtable}{p{0.15\textwidth} p{0.7\textwidth}}
  \toprule
  \textbf{Tool} & \textbf{Description \& Reasoning}
  \\\midrule\midrule
  Go~\cite{noauthor_go_nodate}
  & Go was chosen because of its simple syntax, high performance, strong standard library and third party packages for interacting with Ethereum.
  \\
  go-ipfs-api~\cite{noauthor_go-ipfs-api_2023}
  & A Go package used for interacting with the Kubo implementation of IPFS that gave an easy interface for downloading and uploading data to Kubo.\\
  go-ethereum~\cite{noauthor_go-ethereum_nodate}
  & A collection of tools used for interacting with Ethereum including an Ethereum CLI client Geth, and a tool for converting Ethereum contracts into Go packages.\\
  Zap~\cite{noauthor_zap_2023}
  & A logging library that is much faster than the standard library implementation and has better customisation. \\
  Viper~\cite{noauthor_viper_nodate}
  & A configuration file management library that helps read, write, and access configuration options written to file.
  \\\bottomrule\bottomrule
  \caption{The tools used to develop the backend}
\end{longtable}

\section{Smart Contract}

To write and deploy a smart contract that met the criteria specified in Section~\ref{subsec:design-con-eth}, I used the following tools:

\begin{longtable}{p{0.15\textwidth} p{0.7\textwidth}}
  \toprule
  \textbf{Tool} & \textbf{Description \& Reasoning}
  \\\midrule\midrule
  Solidity~\cite{noauthor_solidity_nodate}
  & The language used to write smart contracts for the Ethereum blockchain.\\
  Sepolia~\cite{noauthor_sepolia_nodate}
  & An Ethereum test-net that used to deploy my smart contract to. One of the main benefits was that it provides a fast transaction time for quick feedback.\\
  Alchemy~\cite{noauthor_alchemy_nodate}
  & Alchemy provides useful tools for interacting with Ethereum and specifically Sepolia, such as an ETH faucet and an RPC URL.\\
  MetaMask~\cite{noauthor_crypto_nodate}
  & A browser-based wallet that can easily be conencted to other tools such as Alchemy or Remix.\\
  Remix~\cite{noauthor_remix_nodate}
  & A browser-based IDE for writing smart contracts that allows for easy deployment.
  \\\bottomrule\bottomrule
  \caption{The tools used for deployment of my smart contract}
\end{longtable}

\vspace{2mm}\noindent
The contract was successfully deployed~\cite{etherscanio_deployed_nodate} to the Sepolia test-net and can be interacted with by any user.

\section{Frontend}

\begin{longtable}{p{0.15\textwidth} p{0.7\textwidth}}
  \toprule
  \textbf{Tool} & \textbf{Description \& Reasoning}
  \\\midrule\midrule
  Wails~\cite{noauthor_wails_nodate}
  & Allows you to add a webkit frontend to a Go application, so that you can use a modern web framework. This allowed me to easily create a reactive UI using tools I was previously familiar with.

  Wails allows you to implement a controller using functions written in that can be called from the frontend and can emit events that trigger actions in the frontend.
  \\
  Vue.js v3~\cite{noauthor_vuejs_nodate}
  & A reactive, component based web-framework that allows me to create reusable components that react to changes in state and can trigger events at different points in a components lifecycle.
  
  The Vue Router~\cite{noauthor_vue_nodate} package was used to add multiple pages to the application and markdown-it~\cite{noauthor_markdown-it_2023} was used to render markdown files.
  \\
  Pinia~\cite{noauthor_pinia_nodate}
  & A state management tool for Vue.js that boosts the reusabiltiy of components and reduces the overall complexity of the frontend. 
  \\
  SASS~\cite{noauthor_sass_nodate}
  & An extension of CSS that is used to style DOM elements. This was essential in making the UI look nice and be accessible.
  \\
  \bottomrule\bottomrule
\end{longtable}

\section{Other Tools}

The following tools were also used throughout development:

\begin{longtable}{p{0.15\textwidth} p{0.7\textwidth}}
  \toprule
  \textbf{Tool} & \textbf{Description \& Reasoning}
  \\\midrule\midrule
  Git~\cite{noauthor_git_nodate}\newline GitHub~\cite{noauthor_github_nodate}
  & A version control system used in conjunction with GitHub. Creating periodic commits meant I always had a recent backup available and could easily backtrack to help find issues.
  Use of a GitHub Actions helped remind me that not all of my tests passed at all times :(.\\
  LaTeX~\cite{noauthor_latex_nodate}
  & Used for the write-up of this document. LaTeX was useful in creating a large document and has many packages that help with referencing and design.\\
  VSCode~\cite{noauthor_visual_nodate}
  & My code editor of choice for this project as it allowed me to seamlessly work on both my frontend and backend code at once.\\
  Lucidchart~\cite{noauthor_lucidchart_nodate}
  & Lucidchart was used to create all of the diagrams for this project.
  \\\bottomrule\bottomrule
  \caption{General purpose tools used for this project}
\end{longtable}
